\chapwithtoc{Conclusion}
	\red{Here or at the~end of each section. Something about the~future of this work?}
	
	\red{\section*{Notes}}
		\red{General notes about the~thesis:
			\begin{itemize}[topsep=4pt,itemsep=2pt]
				\item Check that all of the~classes and other code are marked the~same way in the~text. I used italics somewhere, could use different font for this instead.
				\item Check unbreakable space in front of articles. Remove excessive article usage with proper nouns.
				\item Currently using margins for single-sided printing (bigger on the left side).
				\item Check that present tense is used
				\item Active vs passive voice usage
				\item American English quotation marks (") instead of British English (').
				\item Some of the overfull hbox warnings might change if duplex printing is used (they generate black rectangles on the~edge of the page), leaving them be for now
				\item Check nobreakdash usage (is it always necessary)
				\item Check capitalized references (e.g., Figure, Section, Equation)
				\item Check \textbackslash(...\textbackslash) math mode instead of \$...\$. (actually unlike \textbackslash[...\textbackslash] math mode, there is apparently no real benefit to this clumsy syntax)
				\item Use siunitx package to ensure correct formatting, physics package for derivatives.
				\item Check other stuff that's written in the MFF UK template. Apparently it has since been updated and there are some differences (check for them).
				\item Check correct subscripts in equation (italics vs no italics)
				\item Consistent bold marking of points/vectors
				\item Correct footnotes (capital letters, etc.).
				\item Might have to mention GeoGebra as per the non-commercial license agreement (Made with GeoGebra®) -- maybe put it into acknowledgments next to the MetaCentrum credit? And list all of the figures where GeoGebra was used?
				\item Maybe make some section outside of References specifically for literature? (such as the old CERN TPC review, ATOMKI review is currently not mentioned, not sure if some Wikipedia articles should get a mention or how do these things work)
				\item Consistent use of bm vs mathbf
				\item Consistent use of $\overbar{\mathcal{M}}$ instead of $\mathcal{M}$ when talking about the map of the means (so most of the time)
				\item Proper equation numbering when deriving a relation
				\item Hugo should be mentioned somewhere in the title pages probably?
				\item Consistent itemize/enumerate style (namely spacing) that looks good (ideally set by some macro? maybe the new MFF UK template will solve this?)
				\item Consistent gas mixture notation (e.g., 90:10 Ar:CO$_2$). Maybe mention at the beginning that it is a~molar ratio.
				\item Labels of figures and tables -- maybe in bold? Abbreviated?
				\item Check graph labels, make them bigger if needed.
				\item "The map" can be viewed as a mapping between spaces or just as a coordinate transform.
				\item Maybe switch to cleveref.
				\item siunitx qty not SI
			\end{itemize}
		Random notes:
			\begin{itemize}[topsep=4pt,itemsep=2pt]
				\item Terminology consistency -- ionization/primary/secondary electrons
				\item Consistent \ac{TPC} vs \ac{OFTPC} acronym usage in the text or individual chapters.
				\item Only electrons that start and end in the~sector closer than 0.5~cm are used for reconstruction (newest version).
				\item Attachment, Penning transfer and secondary ionization not considered in the~microscopic simulation.
				\item Suspicious artifacts of trilinear interpolation in \cref{fig:mag}. \textbf{Fixed -- integers instead of doubles in the implementation, influenced reconstruction SIGNIFICANTLY (but not simulation).}
				\item Profiling of the reconstruction!!!! Find out what's taking the most time (probably Runge-Kutta integration which the fit calls a lot). Could gradually decrease the step size to refine the fit instead of making it small right away (arbitrarily small -- the effect of this was never tested). This could take some time to do properly (find a profiler or make profiling macros).
				\item Slow drift velocity good for $z$ reconstruction, too low leads to recombination
				\item Could add link to the GitHub repository, mention CMake? Details about simulating on MetaCentrum?
				\item The first used track had 8~MeV momentum $p = \gamma m v$ (not kinetic energy $E_\text{kin} = (\gamma-1) m c^2 = \sqrt{p^2c^2+m^2c^4}-mc^2 \approx 7.5$~MeV)
				\item Maybe cite \garfieldpp user manual instead?
				\item Using TRandom3 for random number generation.
				\item Does the RK fit error correlate with the actual error?
				\item Some Garfield settings in micro track generation probably not mentioned
			\end{itemize}
		}
	\section*{Future}
		\red{Things planned for the future:
		\begin{itemize}[topsep=4pt,itemsep=2pt]
			\item Testing the reconstruction algorithm by measuring real particles with a~known energy distribution.
			\item The~\textbf{Fast Simulation with Ionization Electron Map} is planned for the~future. It will use the~\ac{HEED} program~\cite{HEED} to simulate the~primary particle and the~Ionization Electron Map (see Section~\ref{sec:map}) to simulate the~drift of secondary electrons. It should be significantly faster than the~Microscopic Simulation but offer comparable precision since it will rely on an already simulated drift map. (Primary track simulated in HEED. Readout parameters by interpolating the map.	Diffusion from the map for randomization.)
			\item Account for GEM, delta electrons, ...
			\item Likelihood approach instead of least squares (if it improves the~reconstruction significantly), we should at least use a~better method than taking the~center of the~TPC bin.
			\item More detailed electric field simulation (if needed, GEM will have more complex field, some irregularities in the field should be considered)
			\item Account for the triggering in MWPC/TPX3 (particle travels from TPX3 to MWPC basically immediately -- fraction of a~nanosecond so there should be no significant difference)
		\end{itemize}}
		
		\red{\subsection*{Likelihood - inverse map}}
			\red{If we wanted to further improve this procedure, taking into account the whole map $\mathcal{M}$, we could make an "inverse map" from $\mathcal{R}$ to distributions on $\mathcal{D}$. We could achieve this by taking the~normalized probability density of an~electron with initial coordinates $(x,y,z)$ having readout coordinates $(x',y',t)$. If we fix $(x',y',t)$, we get an~unnormalized probability density $f(x,y,z) = \mathcal{M}_{(x,y,z)}(x',y',t)$ (assuming that all initial coordinates are a~priori equally likely). This could potentially improve the~discrete reconstruction if we take the~mean value of this probability density across the pad and time bin
				\begin{equation}
					\color{red}
					f_\text{pad, bin}(x,y,z) = \frac{1}{A_\text{pad} \Delta t_\text{bin}} \int_\text{pad, bin} \mathcal{M}_{(x,y,z)}(x',y',t) 	\text{d}x'\text{d}y'\text{d}t
				\end{equation}
			and using it for a~likelihood fit instead of using least squares. This still assumes that all initial coordinates are equally likely which is clearly not the~case for a~primary particle track. In the future, we could even use the~fast track simulation with the~map (should be possible to make around 1000 tracks per minute per core with current settings), create a~big set of tracks with reasonable parameters and use these to get an~approximation of the probability distribution of the~detector response. Some approximations would be necessary when interpreting the~data to decrease the~degrees of freedom of this distribution (we would have to pick a set of parameters and assume that some of them are independent). This could give us an~idea about the~best achievable resolution (how significantly will the~detector response differ for a~given change in energy). If the~difference is significant, we could try to further improve the~likelihood fit.}