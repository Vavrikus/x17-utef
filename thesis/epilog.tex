\chapwithtoc{Conclusion}
	We have developed and implemented several methods for the reconstruction of electron and positron trajectories inside the Orthogonal Fields Time Projection Chambers (\acp{OFTPC}) that will be used in the X17 project at the \acf{IEAPCTU} to confirm or disprove the ATOMKI anomaly~\cite{atomki_be}. We tested these methods on simulated tracks and made a~preliminary estimate of achievable energy resolution.
	
	We used the \garfieldpp toolkit~\cite{Garfield++} for simulations in combination with the ROOT~framework~\cite{ROOT} for data analysis and visualization. Some of our more demanding simulations were run on the MetaCentrum grid~\cite{metacentrum}. The main method of track simulation was the microscopic simulation (provided by the \texttt{Avalanche\-Microscopic} class), which follows ionization electrons from collision to collision.
	
	\subsubsection*{Track Reconstruction}
		The inhomogeneous magnetic field created by permanent magnets is perpendicular to the electric field of the chambers and has a~significant effect on the drift of ionization electrons (as demonstrated in \cref{sec:rasd}). For this reason, we created the ionization electron map, i.e., a~mapping from the detector space~$\mathcal{D}$ (coordinates $(x,y,z)$), where the initial positions of ionization electrons lie, to the readout space~$\mathcal{R}$ (coordinates $(x',y',t)$), where their endpoints on the readout plane lie. The map was simulated using the microscopic simulation. One hundred electrons with low initial velocity were simulated for each point on a~Cartesian grid to get their mean readout coordinates and their covariance. Two gas mixtures 90:10 and 70:30 Ar:CO$_2$ were compared. In the former, the drift velocity is significantly higher, leading to increased effect of the magnetic field and larger diffusion (see, for example, \cref{fig:map_yx_bot}).
		
		To use the map in reconstruction, we have to invert it. We implemented two methods -- a~gradient descent algorithm that finds a~minimum in the trilinearly interpolated map, and a~polynomial interpolation on the inverse grid in the readout space (where we know the inverse from the simulation). Both methods reach almost identical result for the testing track, while the latter is much faster, does not require optimization of parameters, and can be accelerated by precalculating the interpolation coefficients in the future if needed. The reconstruction was shown to be accurate with less than \qty{2}{\mm} \acs{FWHM}, although a~slight shift in the $z$\nobreakdash-coordinate was detected when accounting for different initial velocities of the ionization electrons (see \cref{fig:9010_map_res,fig:7030_map_res}).
		
		The discrete reconstruction takes into account the anode segmentation into pads and the finite size of the time bins. It neglects the gaps between the pads and assumes an ideal readout of charge, counting each individual electron hitting the pad. Charge multiplication in the triple-\ac{GEM} stack used in the \ac{OFTPC} is not taken into account. For each pad/time-bin combination, we use the map inversion to reconstruct its center. Reconstruction of detector coordinates corresponding to electrons hitting the pad 12 near the magnet pole (\cref{fig:pad_reco,fig:pad_reco_old}) show that interpolation on the inverse grid behaves pathologically for large distortions, unlike the much slower gradient descent algorithm.\red{~Or maybe there is a~problem in the algorithm??? Actually not quite sure, what causes this behavior.}
		
	\subsubsection*{Energy Reconstruction}
		To reconstruct energy of a~particle in a~\ac{OFTPC}, we assess its curvature in the inhomogeneous magnetic field. Preliminary tests were done using tracks reconstructed with the map with no discretization. At first, we tested a~cubic spline fit, where the radius of curvature can be determined analytically. This approach turned out to be unpractical because of relatively low speed and necessity to create the energy estimate from the radius and magnetic field at different points on the track (see \cref{fig:spline}).
		
		Another tested approach (at first as a~2D version without pads, later in 3D with pads) was the fit with circular arc with smoothly attached lines, which is the shape of trajectory of a~particle crossing a~finite volume with homogeneous magnetic field oriented perpendicularly to the particles velocity vector. This way we get a~single reconstructed radius of curvature, however, since the magnetic field varies greatly along the track inside the \ac{OFTPC} volume, it is not clear, which value of magnetic field to use to calculate the energy. Two simple estimates were implemented:
		\begin{enumerate}[nosep]
			\item using the value at the point, where the track crosses the middle $x$\nobreakdash-coordinate, and
			\item using the average value along the circular arc in the fit,
		\end{enumerate}
		in both cases, only the component perpendicular to the plane of the fit is considered. The three dimensional version of the fit has four free parameters (radius, length of the first line, length of the circular arc, and rotation angle around the first line). In order for the fit to converge correctly in all cases, the parameter bounds have to be set carefully. Disallowing pathological values that are not expected for any real track may cause the MIGRAD algorithm to converge at the bound. The fit was successfully tested on a~sample of Runge-Kutta tracks described in \cref{sec:rktest}; the results in \cref{fig:cfit_rk} show that the average field estimate systematically underestimates the energy of the track, but its \acs{FWHM} is much smaller than that of the middle field estimate.
		
		The best developed method was the Runge-Kutta fit. In a~simplified model, we assume that the previous layers of detectors will provide us with exact initial position and direction of the particle, when entering the \ac{OFTPC}. We can then use 4th\nobreakdash-order Runge-Kutta numerical integration to simulate such a~trajectory for different energies. In order to speed up the fit and ensure convergence of the fit, we use a~corrected estimate from the circle-and-lines fit with average field as a~starting point. We tested this method on a~grid-like sample of microscopic tracks, described in \cref{sec:microgrid}, fitting tracks reconstructed with the discrete map inversion. As shown in \cref{fig:rk_dE}, we can reach energy resolution (\acs{FWHM}) \qty{1.56}{\percent} for electrons (which curve away from the readout, enabling charge spreading across multiple channels) and \qty{1.95}{\percent} for positrons, assuming an ideal charge readout with no noise and no amplification.
	
	\red{\section*{Notes}}
		\red{General notes about the thesis:
			\begin{itemize}[topsep=4pt,itemsep=2pt]
				\item Check that all of the classes and other code are marked the same way in the text. I used italics somewhere, could use different font for this instead.
				\item Check unbreakable space in front of articles. Remove excessive article usage with proper nouns.
				\item Currently using margins for single-sided printing (bigger on the left side).
				\item Check that present tense is used
				\item Active vs passive voice usage
				\item American English quotation marks (") instead of British English (').
				\item Some of the overfull hbox warnings might change if duplex printing is used (they generate black rectangles on the edge of the page), leaving them be for now
				\item Check nobreakdash usage (is it always necessary)
				\item Check capitalized references (e.g., Figure, Section, Equation)
				\item Check \textbackslash(...\textbackslash) math mode instead of \$...\$. (actually unlike \textbackslash[...\textbackslash] math mode, there is apparently no real benefit to this clumsy syntax)
				\item Use siunitx package to ensure correct formatting, physics package for derivatives.
				\item Check other stuff that's written in the MFF UK template. Apparently it has since been updated and there are some differences (check for them).
				\item Check correct subscripts in equation (italics vs no italics)
				\item Consistent bold marking of points/vectors
				\item Correct footnotes (capital letters, etc.).
				\item Might have to mention GeoGebra as per the non-commercial license agreement (Made with GeoGebra®) -- maybe put it into acknowledgments next to the MetaCentrum credit? And list all of the figures where GeoGebra was used?
				\item Maybe make some section outside of References specifically for literature? (such as the old CERN TPC review, ATOMKI review is currently not mentioned, not sure if some Wikipedia articles should get a mention or how do these things work)
				\item Consistent use of bm vs mathbf
				\item Consistent use of $\overbar{\mathcal{M}}$ instead of $\mathcal{M}$ when talking about the map of the means (so most of the time)
				\item Proper equation numbering when deriving a relation
				\item Hugo should be mentioned somewhere in the title pages probably?
				\item Consistent itemize/enumerate style (namely spacing) that looks good (ideally set by some macro? maybe the new MFF UK template will solve this?)
				\item Consistent gas mixture notation (e.g., 90:10 Ar:CO$_2$). Maybe mention at the beginning that it is a~molar ratio.
				\item Labels of figures and tables -- maybe in bold? Abbreviated?
				\item Check graph labels, make them bigger if needed.
				\item "The map" can be viewed as a mapping between spaces or just as a coordinate transform.
				\item Maybe switch to cleveref.
				\item siunitx qty not SI
				\item Correct em dash?
				\item In the future, it might be useful to save pictures as tex files whenever possible and using siunitx for the labels
				\item Should list of figures/tables be in the thesis? Also, it is currently not in TOC.
				\item link to GitHub
			\end{itemize}
		Random notes:
			\begin{itemize}[topsep=4pt,itemsep=2pt]
				\item Terminology consistency -- ionization/primary/secondary electrons
				\item Consistent \ac{TPC} vs \ac{OFTPC} acronym usage in the text or individual chapters.
				\item Only electrons that start and end in the sector closer than 0.5~cm are used for reconstruction (newest version).
				\item Attachment, Penning transfer and secondary ionization not considered in the microscopic simulation.
				\item Suspicious artifacts of trilinear interpolation in \cref{fig:mag}. \textbf{Fixed -- integers instead of doubles in the implementation, influenced reconstruction SIGNIFICANTLY (but not simulation).}
				\item Profiling of the reconstruction!!!! Find out what's taking the most time (probably Runge-Kutta integration which the fit calls a lot). Could gradually decrease the step size to refine the fit instead of making it small right away (arbitrarily small -- the effect of this was never tested). This could take some time to do properly (find a profiler or make profiling macros).
				\item Slow drift velocity good for $z$ reconstruction, too low leads to recombination
				\item Could add link to the GitHub repository, mention CMake? Details about simulating on MetaCentrum?
				\item The first used track had 8~MeV momentum $p = \gamma m v$ (not kinetic energy $E_\text{kin} = (\gamma-1) m c^2 = \sqrt{p^2c^2+m^2c^4}-mc^2 \approx 7.5$~MeV)
				\item Maybe cite \garfieldpp user manual instead?
				\item Using TRandom3 for random number generation.
				\item Does the RK fit error correlate with the actual error?
				\item Some Garfield settings in micro track generation probably not mentioned
				\item one-to-one only means injection (not bijection)? Make sure correct terminology is used.
				\item circle-and-lines dashes?
			\end{itemize}
		}
	\section*{Future}
	\addcontentsline{toc}{section}{Future}
		\red{Should this be in the TOC?}
	
		\red{Things planned for the future:
		\begin{itemize}[topsep=4pt,itemsep=2pt]
			\item Testing the reconstruction algorithm by measuring real particles with a~known energy distribution (at first just laser, muons?).
			\item The \textbf{Fast Simulation with Ionization Electron Map} is planned for the future. It will use the \ac{HEED} program~\cite{HEED} to simulate the primary particle and the Ionization Electron Map (see Section~\ref{sec:map}) to simulate the drift of secondary electrons. It should be significantly faster than the Microscopic Simulation but offer comparable precision since it will rely on an already simulated drift map. (Primary track simulated in HEED. Readout parameters by interpolating the map. Diffusion from the map for randomization.) Currently more or less implemented, but it turns out that the initial energy of ionization electrons cannot be discarded.
			\item Account for GEM, delta electrons, ...
			\item Likelihood approach instead of least squares (if it improves the reconstruction significantly), we should at least use a~better method than taking the center of the TPC bin.
			\item More detailed electric field simulation (if needed, GEM will have more complex field, some irregularities in the field should be considered)
			\item Account for the triggering in MWPC/TPX3 (particle travels from TPX3 to MWPC basically immediately -- fraction of a~nanosecond so there should be no significant difference)
			\item Better choice of initial velocity in the map? Maybe look at the distribution for e+e- with different energies and either use this distribution combined for all (if the dependence isn't too big) or even few different simulations? Might (?) raise the requirement of electrons simulated per position. Or maybe just make some adjustment of the $z$ coordinate in the end?
			\item Create a pad map. Might be worth it to use gradient descent search after all.
			\item Testing of different pad layouts possible.
		\end{itemize}}
		
		\red{\subsection*{Likelihood - inverse map}}
			\red{If we wanted to further improve this procedure, taking into account the whole map $\mathcal{M}$, we could make an "inverse map" from $\mathcal{R}$ to distributions on $\mathcal{D}$. We could achieve this by taking the normalized probability density of an electron with initial coordinates $(x,y,z)$ having readout coordinates $(x',y',t)$. If we fix $(x',y',t)$, we get an unnormalized probability density $f(x,y,z) = \mathcal{M}_{(x,y,z)}(x',y',t)$ (assuming that all initial coordinates are a~priori equally likely). This could potentially improve the discrete reconstruction if we take the mean value of this probability density across the pad and time bin
				\begin{equation}
					\color{red}
					f_\text{pad, bin}(x,y,z) = \frac{1}{A_\text{pad} \Delta t_\text{bin}} \int_\text{pad, bin} \mathcal{M}_{(x,y,z)}(x',y',t) 	\text{d}x'\text{d}y'\text{d}t
				\end{equation}
			and using it for a~likelihood fit instead of using least squares. This still assumes that all initial coordinates are equally likely which is clearly not the case for a~primary particle track. In the future, we could even use the fast track simulation with the map (should be possible to make around 1000 tracks per minute per core with current settings), create a~big set of tracks with reasonable parameters and use these to get an approximation of the probability distribution of the detector response. Some approximations would be necessary when interpreting the data to decrease the degrees of freedom of this distribution (we would have to pick a set of parameters and assume that some of them are independent). This could give us an idea about the best achievable resolution (how significantly will the detector response differ for a~given change in energy). If the difference is significant, we could try to further improve the likelihood fit.}