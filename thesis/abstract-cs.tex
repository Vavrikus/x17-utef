%%% A template of a stand-alone abstract in Czech

% Meta-data of your thesis (please edit)
\input metadata.tex

% Generate metadata in XMP format for use by the pdfx package
\let\OrigThesisTitleXMP=\ThesisTitleXMP
\def\ThesisTitleXMP{\OrigThesisTitleXMP\space (abstrakt)}
\def\AbstractXMP{}
\def\ThesisKeywordsXMP{}
\input xmp.tex

\documentclass[12pt]{report}

\usepackage[a4paper, hmargin=1in, vmargin=1in]{geometry}
\usepackage[a-2u]{pdfx}
\usepackage[czech]{babel}
\usepackage{iftex}
\ifpdftex
\usepackage[utf8]{inputenc}
\usepackage[T1]{fontenc}
\usepackage{textcomp}
\fi
\usepackage{lmodern}
\usepackage{amsmath}
\usepackage{amsthm}
\usepackage{amsfonts}
\usepackage{fancyvrb}
\usepackage{jabbrv}
\usepackage{siunitx}

\pagenumbering{gobble}

% Definitions of macros (see description inside)
\input macros.tex

\begin{document}

\ifx\StudyLanguage\LangCS

% By default, we create the abstract from the metadata
V této práci byl vyvinut rekonstrukční algoritmus pro atypickou časovou projekční komoru (TPC), jež bude použita na ÚTEF ČVUT k~hledání anomálie X17 ve vnitřní kreaci párů. Tyto komory mají nehomogenní toroidální magnetické pole, orientované kolmo k~elektrickému; proto je nazýváme TPC s~ortogonálními poli (OFTPC). Toto uspořádání způsobuje distorzi driftu v~komoře a~komplikuje tvar elektronových a~pozitronových tracků. Představujeme několik řešení těchto komplikací, z~nichž nejlepší využívá simulaci mapy ionizačních elektronů pro rekonstrukci tracků a~Runge-Kutta fit pro rekonstrukci energie. Na závěr pomocí simulací demonstrujeme rozlišení (FWHM) \qty{1.6}{\percent} pro elektrony a~\qty{2.0}{\percent} pro pozitrony, za předpokladu ideálního vyčítání náboje bez zesílení a~šumu, a~bez korekce systematických odchylek závislých na parametrech tracků.

\else

Students of English study programmes do not submit a~Czech version
of the abstract.

\fi

\end{document}
