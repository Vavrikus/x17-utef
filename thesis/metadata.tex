%%% Please fill in basic information on your thesis, which will be automatically
%%% inserted at the right places. You need to replace \xxx{...} by real data.

% Type of your thesis:
%	"bc" for Bachelor's
%	"mgr" for Master's
%	"phd" for PhD
%	"rig" for rigorosum
\def\ThesisType{bc}

% Language of your study programme:
%	"cs" for Czech
%	"en" for English
\def\StudyLanguage{cs}

% Thesis title in English (exactly as in the formal assignment)
\def\ThesisTitle{Simulation and Reconstruction of~Charged Particle Trajectories in~an Aty\-pic Time Projection Chamber}

% Author of the thesis (you)
\def\ThesisAuthor{Martin Vavřík}

% Year when the thesis is submitted
\def\YearSubmitted{2025}

% Name of the department or institute, where the work was officially assigned
% (according to the Organizational Structure of MFF UK in English,
% see https://www.mff.cuni.cz/en/faculty/organizational-structure,
% or a full name of a department outside MFF)
\def\Department{Institute of Particle and Nuclear Physics}

% Is it a department (katedra), or an institute (ústav)?
\def\DeptType{Institute}

% Thesis supervisor: name, surname and titles
\def\Supervisor{Mgr. Tomáš Sýkora, Ph.D.}

% Supervisor's department (again according to Organizational structure of MFF)
\def\SupervisorsDepartment{Institute of Particle and Nuclear Physics}

% Study programme and specialization
\def\StudyProgramme{Physics}
%%% OLD??? \def\StudyBranch{Physics}

% An optional dedication: you can thank whomever you wish (your supervisor,
% consultant, a person who lent the software, etc.)
\def\Dedication{%
	Dedication.
}

% Abstract (recommended length around 80-200 words, try <= 120; this is not a copy of your thesis assignment!)
\def\Abstract{%
	In this work, we describe the development of a~reconstruction algorithm for atypical \acfp{TPC} that will be used at \acs{IEAPCTU} to search for the ATOMKI anomalous internal pair creation~\cite{atomki_be}. The chambers have an inhomogeneous toroidal magnetic field orthogonal to their electric field, hence we call them \acfp{OFTPC}. This arrangement causes a~distortion of the drift inside the chamber and complicates the shape of electron/positron trajectories. We present a~few approaches to tackle these problems, the best of which uses a~simulated ionization electron drift map for track reconstruction, and Runge-Kutta fit for energy reconstruction. Finally, we show from simulations that, for an ideal charge readout with no multiplication and no noise, we can reach an energy resolution .......
}

% 3 to 5 keywords (recommended) separated by \sep
% Keywords are useful for indexing and searching for the theses by topic.
\def\ThesisKeywords{%
	simulation \sep reconstruction \sep time projection chamber
}

% If any of your metadata strings contains TeX macros, you need to provide
% a plain-text version for use in XMP metadata embedded in the output PDF file.
% If you are not sure, check the generated thesis.xmpdata file.
\def\ThesisAuthorXMP{\ThesisAuthor}
\def\ThesisTitleXMP{\ThesisTitle}
\def\ThesisKeywordsXMP{\ThesisKeywords}
\def\AbstractXMP{chybííííí}

% If your abstracts are long and do not fit in the infopage, you can make the
% fonts a bit smaller by this setting. (Also, you should try to compress your abstract more.)
\def\InfoPageFont{}
%\def\InfoPageFont{\small}  % uncomment to decrease font size

% If you are studing in a Czech programme, you also need to provide metadata in Czech:
% (in English programmes, this is not used anywhere)

\def\ThesisTitleCS{Simulace a rekonstrukce drah nabitých částic v atypické časově projekční komoře}
\def\DepartmentCS{Ústav částicové a jaderné fyziky}
\def\DeptTypeCS{Ústav}
\def\SupervisorsDepartmentCS{Ústav částicové a jaderné fyziky}
\def\StudyProgrammeCS{Fyzika}

\def\ThesisKeywordsCS{%
    simulace \sep rekonstrukce \sep časová projekční komora
}

\def\AbstractCS{%
    Abstrakt práce přeložte také do češtiny!!!!!!!!!!!!!!!!!!!!!!!!!!!!!!!!!!!!!!!!!!!!!!
}