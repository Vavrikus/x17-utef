\chapter{Energy Reconstruction}
\label{sec:energy}
	The~second stage is the reconstruction of the particle's energy using a~fit of its reconstructed track (see Section~\ref{sec:track}). We have tested three ways of reconstructing the energy. Fitting is done using the MINUIT algorithm implemented in ROOT~\cite{ROOT}.\red{~Cite some CERN article directly on MINUIT, can add a~section. Or is it done using MIGRAD? The circle and RK4 probably was.}
	
	The~\textbf{Cubic Spline Fit} was a~tested and later rejected method of energy reconstruction. It uses smoothly connected piecewise cubic polynomials between uniformly spaced nodes. The reconstructed energy is calculated using the fit parameters by computing the radius of curvature in different points of the fitted curve using the known magnitude of the magnetic field perpendicular to the trajectory. We rejected this method because the tuning of the fit turned out to be unpractical compared to the other used methods.\red{~Reconstructs energy at every position (even though the actual energy doesn't change much) and it might be slower but no profiling has been done yet. Of course, it wasn't tested on the newer track reconstruction methods at all.}
	
	The~\textbf{Circle and Lines Fit} was chosen as an~alternative since this corresponds to the shape of a~trajectory of a~charged particle crossing a~finite volume with a~homogeneous magnetic field. The~energy of the particle can be estimated using the fitted radius and the magnitude of the perpendicular magnetic field in the middle of the \ac{TPC}.
	
	The~\textbf{Runge-Kutta Fit} uses the 4th order Runge-Kutta numerical integration described in Section~\ref{sec:rks}. Initial parameters of the track (including the particle's energy) are optimized so that the integrated trajectory fits to the reconstructed one. This fit can also be performed as a~single parameter (i.e., energy) fit if we get the initial position and orientation of the particle on the entrance to the \ac{TPC} from previous detectors (\ac{TPX3} and \ac{MWPC}, see Section~\ref{sec:IEAP}).
	
	\begin{figure}
		\centering
		\includegraphics[width=0.5\textwidth]{9010_3d.png}
		\caption{Example of a~fitted reconstructed track.\red{~Swap for better image.}}
		\label{fig:90103d}
	\end{figure}
	
	\section{Cubic Spline Fit}
	\label{sec:cspline}
		The~first method for the estimation of the kinetic energy of the particle uses a~cubic spline fit. We use an~electron track simulated using the microscopic simulation, described in detail in Section~\ref{sec:microfirst}. The track was reconstructed using the map described in Section~\ref{sec:map}.
				
		In order to calculate the spline, we use the class \textit{TSpline3} from ROOT. This allows us to evaluate the spline using the coordinates $(x_n,z_n)$ of each node and the derivatives $d_1,d_2$ in the first and the last node. We can fit these parameters of a~fixed amount of nodes to the simulated trajectory. We use the IMPROVE algorithm provided by the \textit{TMinuit} class in ROOT\orange{~(there are some guidelines for fonts in MFF UK template (Czech version) that I will eventually apply (see notes in the conclusion))}. This algorithm attempts to find a~better local minimum after converging\orange{~(could reformulate a~bit, taken word for word from some manual)}.
		
		After the fit converges, we calculate an~energy estimate using the radius of curvature, which we can extract from the fitted spline equation at every point of the trajectory. The~part of the spline corresponding to a~given node is defined as
			\begin{equation}
				z(x) = z_n + b \Delta x+c(\Delta x)^2+d(\Delta x)^3,
			\end{equation}
		where $\Delta x = x-x_n$ and $b,c,d$ are coefficients. Using this equation, we derive the radius of curvature\footnote{For the general formula see \url{https://en.wikipedia.org/wiki/Curvature\#Graph_of_a_function}.} as:
			\begin{equation}
				r(x) = \frac{\left(1+z'^2(x)\right)^\frac{3}{2}}{z''(x)} = \frac{\left(1+\left(b+2c\Delta x+3d(\Delta x)^2\right)^2\right)^\frac{3}{2}}{2c+6d\Delta x}.
			\end{equation}
		Based on the geometry of our~detector, we assume that the magnetic field satisfies $\bm{B}(x,0,z) = (0,B(x,z),0)$ for a~track in the XZ~plane. Since the electron is relativistic, the effect of the electric field on its trajectory is negligible. The~Lorentz force $F_L$ is then always perpendicular to the momentum of the electron and acts as a~centripetal force $F_c$\orange{~(not quite sure how to handle this then?)}:
			\begin{gather}
				\begin{aligned}
					\bm{F_L} &= \bm{F_c},\\
					\norm{e\bm{v}\times\bm{B}} &= \frac{\gamma m_e v^2}{r},\\
					e c\beta B &= \frac{E_{0e} \beta^2}{r\sqrt{1-\beta^2}},\\
					\sqrt{1-\beta^2} &= \frac{E_{0e} \beta}{ecBr},
				\end{aligned}\\
				\beta^2(x) = \left[1+\left(\frac{E_{0e}}{ecB(x,z(x))r(x)}\right)^2\right]^{-1}, \label{eq:ekin1}
			\end{gather}
		where $e$~is the elementary charge, $c$~is the speed of light in vacuum, $m_e$~is the rest mass of electron, $E_{0e} = m_e c^2$ is its rest energy, $\gamma$~is the Lorentz factor, $\bm{v}$~is the velocity of the electron, and $\beta = \frac{v}{c}$. The~kinetic energy for a~given point on the trajectory is then given as
			\begin{equation}
				\label{eq:ekin2}
				E_\text{kin}(x) = \left(\frac{1}{\sqrt{1-\beta^2(x)}}-1\right)E_{0e}.
			\end{equation}
		We can then average these estimates at multiple points (\red{possibly using some weights to account for the change in accuracy, }this wasn't optimized and we just ended with the graph) to get a~single value. This method was later rejected in favor of the circle and lines fit\orange{~(the name was already established at the beginning of the chapter)} described in the next section.
		\red{Add some figures.}
		
		\begin{figure}
			\centering
			\includegraphics[width=0.8\textwidth]{9010_splines.png}
			\caption{First attempt at a~track reconstruction using only the drift velocity. Spline energy reconstruction attempt.\red{~Swap for better image(s) -- subfigure environment, correct coordinates.}}
			\label{fig:9010splines}
		\end{figure}
	
	\section{Circle and Lines Fit}
	\label{sec:clines}
		Another way to estimate the particle's kinetic energy is to~fit its\orange{~(??)} trajectory with a~circular arc with lines attached smoothly. This shape of trajectory corresponds to a~movement of a~charged particle through a~homogeneous magnetic field perpendicular to the particle's momentum and limited to a~certain volume. In general, the shape of such a~trajectory with a~non-perpendicularly oriented momentum is a~spiral. In our case, the magnetic field is approximately toroidal and the particle motion is nearly perpendicular to it\red{~(verify, could add some magnetic field plots in different vertical planes; shouldn't have a big effect on the reconstructed radius anyway)}. At first, we tested a~2D version of this fit, then we adapted it to 3D.
		
		The~field in our detector is not homogeneous, it is therefore not entirely clear what value of magnetic field should be used along with the fitted radius (using equations~\ref{eq:ekin1} and~\ref{eq:ekin2}) to get the best estimate for the kinetic energy. Since we only use this method as the first iteration of the particle's energy that we later refine, an~optimal solution of this problem is not required. Instead, we tested two options: taking the value of the field in the middle of the fitted circular arc\red{~(or is it in the middle $x$ of the OFTPC?)} and taking the average field along it.\red{~We haven't really tried to plot this for multiple tracks, but these estimates are saved somewhere and could be plotted.}
		
		\subsection{Two-dimensional fit}
			In the 2D case, the fitted function used for the electron track\footnote{Electron tracks bend towards negative~$z$, we need to use the upper part of the circle.} described in Section~\ref{sec:microfirst}\red{~(one specific track at the time, technically this function doesn't work for a curvature that gets outside of the semicircle)} is defined as follows:
				\begin{equation}
					\label{eq:clines2d}
					z(x) = \begin{cases}
								a_1x+b_1 & x<x_1\\
								z_0+\sqrt{r^2-(x-x_0)^2} & x_1\leq x\leq x_2\\
								a_2x+b_2 & x>x_2
						   \end{cases},
				\end{equation}
			where $a_{1,2}$ and $b_{1,2}$ are the parameters of the lines, $(x_0,z_0)$ is the center of the circle, $r$ is its radius, and $(x_{1,2},z_{1,2})$ are the coordinates of the function's nodes. That means we have 9~parameters ($z_{1,2}$ are not used in the function) along with 2~continuity conditions and 2~smoothness conditions\orange{~(9~parameters of the described function, 5~of them independent after taking the conditions into account)}. For the fit, we use the coordinates of the nodes and the radius of the circle, which gives us 5~independent parameters (only the radius has to be larger than half of the distance between nodes). The~continuity conditions (combined with the relations for $z_{1,2}$) are
				\begin{equation}
					\label{eq:ccont}
					z_{1,2} = a_{1,2}x_{1,2}+b_{1,2} = z_0-\sqrt{r^2-(x_{1,2}-x_0)^2},
				\end{equation}
			the smoothness conditions are
				\begin{equation}
					\label{eq:a12}
					a_{1,2} = \frac{x_0-x_{1,2}}{\sqrt{r^2-(x_{1,2}-x_0)^2}}.
				\end{equation}
			Together with the Equation~\ref{eq:ccont} we get the values of $b_{1,2}$
				\begin{equation}
					\label{eq:b12}
					b_{1,2} = z_{1,2} - a_{1,2} x_{1,2}.
				\end{equation}
			For the coordinates of the center of the circle, we can use the fact that the center has to lie on the axis of its chord. In other words, there is a~value of a~parameter~$t$ such that, using the parametric equation of the axis
				\begin{equation}
					\begin{pmatrix} x_0\\ z_0 \end{pmatrix} = \begin{pmatrix} \frac{x_1+x_2}{2}\\ \frac{z_1+z_2}{2} \end{pmatrix} + t \begin{pmatrix} \frac{z_2-z_1}{2}\\ \frac{x_1-x_2}{2} \end{pmatrix}.
				\end{equation}
			At the same time, the center has to be in a~distance of $r$ from the nodes:
				\begin{gather}
					(x_1-x_0)^2 + (z_1-z_0)^2 = r^2,\notag\\
					\left(\frac{x_1-x_2}{2}+\frac{z_1-z_2}{2}t\right)^2 + \left(\frac{z_1-z_2}{2}+\frac{x_2-x_1}{2}t\right)^2 = r^2,\notag\\
					\left(\left(\frac{x_2-x_1}{2}\right)^2+\left(\frac{z_2-z_1}{2}\right)^2\right)t^2+\left(\frac{x_2-x_1}{2}\right)^2+\left(\frac{z_2-z_1}{2}\right)^2-r^2=0.
				\end{gather}
			Since our electron track bends towards negative $z$ and $x_2 > x_1$, we only care about the solution with $t>0$
				\begin{gather}
					t = \sqrt{\frac{r^2}{\left(\frac{x_2-x_1}{2}\right)^2+\left(\frac{z_2-z_1}{2}\right)^2}-1},\\
					\begin{aligned}
						x_0 = \frac{x_1+x_2}{2} + \frac{z_2-z_1}{2} \sqrt{\frac{r^2}{\left(\frac{x_2-x_1}{2}\right)^2+\left(\frac{z_2-z_1}{2}\right)^2}-1},\label{eq:xz0}\\
						z_0 = \frac{z_1+z_2}{2} - \frac{x_2-x_1}{2} \sqrt{\frac{r^2}{\left(\frac{x_2-x_1}{2}\right)^2+\left(\frac{z_2-z_1}{2}\right)^2}-1}.
					\end{aligned}
				\end{gather}
			The~function defined in Equation~\ref{eq:clines2d} along with equations~\ref{eq:a12}, \ref{eq:b12}, and \ref{eq:xz0} derived using the continuity and smoothness conditions (combined with the relations for $z_{1,2}$) fully define our fitted function with parameters $r,x_{1,2},z_{1,2}$.\red{~Some pictures of the fit on the tested track. Results of the fit. Again, the actual fit uses 8-z. Use GeoGebra schematics to generate a~picture of 2D geometry.}
			
			\begin{figure}
				\centering
				\includegraphics[width=0.8\textwidth]{9010_circle2D.png}
				\caption{First attempt at a~track reconstruction using only the drift velocity. Circle and Lines Fit in 2D.\red{~Swap for better image, correct coordinates.}\orange{~Bias should be described in the previous chapter, not here.}}
				\label{fig:9010circle2D}
			\end{figure}
		
		\subsection{Three-dimensional fit}
			In three dimensions, the shape of a~trajectory of a~charged particle in a~uniform magnetic field is a~cylindrical helix. Nevertheless, since we assume that the field is approximately perpendicular to the particle's momentum at all times, we will further approximate the trajectory with a~circular arc $\mathbf{X}_\text{C}(\phi)$ (with lines $\mathbf{X}_\text{L1}(t),\mathbf{X}_\text{L2}(s)$ attached smoothly).
			
			We assume that the initial position $\mathbf{X}_0 = (x_0,y_0,z_0)$ and direction $\theta,\varphi$\red{~(spherical angles as in Section~\ref{sec:coor})} are known, since this information will be provided by \ac{TPX3} and \ac{MWPC} layers.\red{~We could further refine it at the end of the current algorithm with some kind of global fit (all detector layers).} The fit then has four free parameters (see \cref{fig:circle3d}):
				\begin{itemize}[nosep]
					\item the length of the first line $l$ (as measured from the initial position),
					\item the radius of the circular arc $r$,
					\item the central angle of the arc $\phi_\text{max} \in [0,2\pi]$,
					\item the direction of the curvature given by the angle $\alpha \in [0,2\pi]$ (right-handed with respect to the particle direction, $\alpha = 0$ if the particle curves towards negative~$z$ in a~plane given by~$\hat{z}$ and the direction vector).
				\end{itemize}
			\begin{figure}
				\centering
				\includegraphics[width=0.8\textwidth]{circle3d.png}
				\caption{Visualization of the 3D geometry of the Circle and Lines Fit and its parameters.}
				\label{fig:circle3d}
			\end{figure}
			Using these parameters, we can derive a parametrization of the whole curve. Let $\mathbf{v}$ be the initial unit direction vector, i.e., using the spherical angles
				\begin{equation}
					\mathbf{v} = (\cos\varphi\cos\theta, \,\sin\varphi\cos\theta, \,\sin\theta)^\mathrm{T},
				\end{equation}
			then we can parameterize the first line as follows:
				\begin{equation}
					\mathbf{X}_\text{L1}(t) = \mathbf{X}_0 + t\mathbf{v} \quad t\in[0,l].
				\end{equation}
			This gives us the starting point of the arc
				\begin{equation}
					\mathbf{X}_1 = \mathbf{X}_\text{L1}(l) = \mathbf{X}_0 + l\mathbf{v}.
				\end{equation}
			The vector $\mathbf{c}_1$ that lies in the plane of curvature and points from $\mathbf{X}_1$ to the center of curvature can be calculated using a composition of rotations. First, we rotate $\mathbf{v}$ to point in the $\mathbf{\hat{x}}$ direction, the normal for $\alpha = 0$ than points in the $-\mathbf{\hat{z}}$ direction, we apply the $\alpha$ rotation and reverse the rotations into the $\mathbf{\hat{x}}$ direction:\orange{~(parameters are explained in the bullet points above)}
				\begin{equation}
					\begin{aligned}
						\mathbf{c}_1 &= R_z(\varphi)R_y(-\theta)R_x(\alpha)R_y\left(\frac{\pi}{2}\right)R_y(\theta)R_z(-\varphi)\mathbf{v},\\
						&= R_z(\varphi)R_y(-\theta)R_x(\alpha)(-\mathbf{\hat{z}}),\\
						&= \scalebox{0.95}{$
								\begin{pmatrix}
									\cos\varphi & -\sin\varphi & 0\\
									\sin\varphi & \cos\varphi & 0\\
									0 & 0 & 1
								\end{pmatrix}
								\begin{pmatrix}
									\cos\theta & 0 & -\sin\theta\\
									0 & 1 & 0\\
									\sin\theta & 0 & \cos\theta
								\end{pmatrix}
								\begin{pmatrix}
									1 & 0 & 0\\
									0 & \cos\alpha & -\sin\alpha\\
									0 & \sin\alpha & \cos\alpha
								\end{pmatrix}
								\begin{pmatrix}
									0\\ 0\\ -1
								\end{pmatrix}
							$},\\
						&= 	\begin{pmatrix}
								-\sin\alpha\sin\varphi+\cos\alpha\cos\varphi\sin\theta\\
								\phantom{-}\sin\alpha\cos\varphi+\cos\alpha\sin\varphi\sin\theta\\
								-\cos\alpha\cos\theta
							\end{pmatrix}.
					\end{aligned}
				\end{equation}
			\orange{Signs should be correct because right-handed rotation around $y$ rotates $z$ into $x$ and this one is the opposite.}\red{~Seems like in this part of the code $\theta$ is actually taken from the pole. Instead of the equator plane.} Similarly by rotating $\mathbf{\hat{y}}$, we can get the normal vector $\mathbf{n}=\mathbf{v}\cross\mathbf{c}_1$ perpendicular to the plane of the trajectory:
				\begin{equation}
					\mathbf{n} = R_z(\varphi)R_y(-\theta)R_x(\alpha)\mathbf{\hat{y}}=
									\begin{pmatrix}
										-\cos\alpha\sin\varphi-\sin\alpha\cos\varphi\sin\theta\\
										\phantom{-}\cos\alpha\cos\varphi-\sin\alpha\sin\varphi\sin\theta\\
										\sin\alpha\cos\theta
									\end{pmatrix}.
				\end{equation}
			This allows us to express the coordinates of the center $\mathbf{C}$ of the circular arc:
				\begin{equation}
					\mathbf{C} = \mathbf{X}_1+r\mathbf{c}_1.
				\end{equation}
			We can then get the parametrization and the endpoint of the circular arc using Rodrigues' rotation formula:\orange{~(all parameters explained in the bullet points above)}
				\begin{gather}
					\begin{aligned}
						\mathbf{c}_2 &= \mathbf{c}_1\cos\phi_\text{max} + (\mathbf{n}\cross\mathbf{c}_1)\sin\phi_\text{max} + \mathbf{n}(\mathbf{n}\cdot\mathbf{c}_1)(1-\cos\phi_\text{max}),\\
						&= \mathbf{c}_1\cos\phi_\text{max} - \mathbf{v}\sin\phi_\text{max},
					\end{aligned}\\
					\mathbf{X}_\text{C}(\phi) = \mathbf{C} - r(\mathbf{c}_1\cos\phi - \mathbf{v}\sin\phi) \quad \phi\in[0,\phi_\text{max}],\\
					\mathbf{X}_2 = \mathbf{X}_\text{C}(\phi_\text{max}) = \mathbf{C} - r\mathbf{c}_2,
				\end{gather}
			and if we define the direction vector of the second line, we also get its parametrization
				\begin{gather}
					\mathbf{w} = \mathbf{v}\cos\phi_\text{max} + (\mathbf{n}\cross\mathbf{v})\sin\phi_\text{max} = \mathbf{v}\cos\phi_\text{max} + \mathbf{c}_1\sin\phi_\text{max},\\
					\mathbf{X}_\text{L2}(s) = \mathbf{X}_2 + s\mathbf{w} \quad s\in[0,\infty).
				\end{gather}
				
			The fit is performed as a~(weighted) least square minimization\red{~(MIGRAD ROOT)}, therefore we need to derive the distance of any point~$\mathbf{P}$ to the fitted curve. For the first line, we simply compute the parameter value of the closest point on the line:
				\begin{equation}
					\label{eq:segdist}
					\begin{aligned}
						t_P &= \mathbf{v}\cdot(\mathbf{P}-\mathbf{X}_1),\\
						d_{P1} &= \norm{\mathbf{P}-\mathbf{X}_\text{L1}(t_P)}.
					\end{aligned}
				\end{equation}
			If the parameter value is outside of its bounds defined above, we take the boundary value instead. The distance to the second line is computed likewise. For the circular arc\orange{~(specific circular arc in the fit)}, we find the closest point\orange{~(on the arc)} by projecting the center connecting line onto the arc plane:
				\begin{gather}
					\mathbf{X}_{PC} = \mathbf{C} + r\frac{(\mathbf{P}-\mathbf{C})-(\mathbf{n}\cdot(\mathbf{P}-\mathbf{C}))\mathbf{n}}{\norm{(\mathbf{P}-\mathbf{C})-(\mathbf{n}\cdot(\mathbf{P}-\mathbf{C}))\mathbf{n}}},\\
					d_{PC} = \norm{\mathbf{P}-\mathbf{X}_{PC}}
				\end{gather}
			 If the point $\mathbf{X}_{PC}$ lies outside of the arc, distance to the closest endpoint is taken instead. The~shortest distance out of $d_{P1},d_{PC},d_{P2}$ is then taken as the distance to the curve.\red{~When calculating energy with the average field, only the arc is considered. Middle field in the current implementation taken in the middle~$x$ plane (intersection with the curve). TVirtualFitter+MIGRAD, maximal num of iterations, toleration. Different uncertainties in $x,y,z$ not taken into account.}
			
			\red{Fit details (parameter bounds, initial setting).}
			
			\subsection{Testing on a~Runge-Kutta sample}
				The~three dimensional circle and lines fit was tested on a~sample of Runge-Kutta tracks with randomized parameters described in Section~\ref{sec:rktest}. These tracks of primary electrons and positrons consist of points calculated with the \ac{RK4} algorithm for a~given proper time step\red{~(step can be adjusted by dividing by the gamma factor $\rightarrow$ detector time)}.\red{~Fitting with circle only was also partially implemented (didn't work but could be fixed/tuned).}
	
	\section{Runge-Kutta Fit}
	\label{sec:rkfit}
		The~Runge-Kutta fit uses the \acf{RK4} numerical integration of the equation of motion (see Section~\ref{sec:rks}) to find the best values of the track parameters -- the track origin, initial velocity direction and the kinetic energy. In order to speed up the energy reconstruction, an~initial guess of these parameters can be obtained from the 3D circle fit described in the previous section. Furthermore, assuming we know the track origin and orientation, we can perform a~single parameter fit of the kinetic energy\red{~(do some profiling and show that it is faster -- below in the microscopic testing)}.
		
		The~fit is performed as a~least square minimization of the (weighted) distances of the track points (true ionization vertices from the simulation or reconstructed points). The simulated \ac{RK4}~track consists of line segments with known endpoints, therefore we can calculate the distance of a~point from this segment analogically to Equation~\ref{eq:segdist} with $\mathbf{v}$ given as a~unit vector in the direction of the segment.
		
		We need to find the segment with the lowest distance. We assume, that the distance $d_\mathbf{P}(\tau)$ of a~point $\mathbf{P}$ to the point on the track (a~curve parameterized by the proper time $\tau$) $\mathbf{X}(\tau)$ has a~single minimum (local and global), no local maximum (except the interval endpoints) and no~saddle point
			\begin{equation}
				\label{eq:rk_assum}
				\exists!\tau_\text{min}\in[0,\tau_N]\colon\ \left(\forall\tau\in[0,\tau_N]\colon  d_{\mathbf{P}}(\tau) \geq d_{\mathbf{P}}(\tau_\text{min})\right)\ \lor\ \dv{d_{\mathbf{P}}}{\tau}{(\tau_\text{min})} = 0,
			\end{equation}
		where $N$ is the number of \ac{RK4} steps. This is a~reasonable assumption for a~track with an~approximate shape of a~circular arc with a~radius $r$, since the distance $d$ from a~point $\mathbf{C}$ on the corresponding circle of a~point $\mathbf{P}$ offset by~$a$ from the arc plane and by~$b$ from the arc's center when projected on its plane is given by the law of cosines:
			\begin{equation}
				\label{eq:rkdemo}
				d^2 = a^2+b^2+r^2 - 2br\cos\alpha,
			\end{equation}
		where $\alpha$ is the angle between points~$\mathbf{C}$ and~$\mathbf{P}$ as seen from the center of the arc (see \cref{fig:rkdemo}). This function is strictly convex for $\alpha\in\left(-\frac{\pi}{2},\frac{\pi}{2}\right)$ and in our case, the center of the arc lies outside of the detector and $\alpha$ is restricted to a~small interval around zero\red{~(especially considering that the initial guess should make the fitted trajectory reasonably close to any relevant point, in the worst-case scenario, the distance is overestimated which should keep the fit from converging to such solutions)}.
		
		\begin{figure}
			\centering
			\begin{subfigure}[t]{0.7\textwidth}
				\centering
				\includegraphics[width=\textwidth]{rk_circle_demo.png}
			\end{subfigure}
			\hfill
			\begin{subfigure}[t]{0.29\textwidth}
				\centering
				\includegraphics[width=\textwidth]{rk_circle_demo2.png}
			\end{subfigure}
			\caption{Demonstration of the convexity of the distance function $d(\alpha)$ for a~circular track (see Equation~\ref{eq:rkdemo}).}
			\label{fig:rkdemo}
		\end{figure}
		
		In a~more general case, if we consider the vector $\mathbf{a}(\tau) = \mathbf{P}-\mathbf{X}(\tau)$ whose size is $\norm{\mathbf{a}(\tau)} = d_\mathbf{P}(\tau)$, then the we get
			\begin{equation}
				2d_{\mathbf{P}}\dv{d_{\mathbf{P}}}{\tau}= \dv{d^2_{\mathbf{P}}}{\tau} = 2\mathbf{a}\cdot\dv{\mathbf{a}}{\tau} = -2\mathbf{a}\cdot\dv{\mathbf{X}}{\tau},
			\end{equation}
		therefore for the derivative of~$d_\mathbf{P}(\tau)$ to be zero, $\mathbf{a}(\tau)$ has to be perpendicular to the tangent of the track. In 3D, for a~given $\mathbf{X}(\tau)$, this condition restricts $\mathbf{P}$ to a~plane. This means that on a~curving track, for any two points $\mathbf{X}(\tau_1),\mathbf{X}(\tau_2)$ with non-parallel tangents, we can find a~point~$\mathbf{P}$ that has $\dv{d_{\mathbf{P}}}{\tau}{(\tau_1)} = \dv{d_{\mathbf{P}}}{\tau}{(\tau_2)} = 0$, which violates the assumption~\ref{eq:rk_assum}. If we have a~circle-and-lines track as described in the previous sections, such a~point has to lie outside of the circular sector given by the arc.
		
		For a~planar track $\mathbf{X}(\tau) = \left(X_1(\tau),X_2(\tau)\right)$, the envelope of all its normals is the evolute of the curve (i.e., the set of centers of all its osculating circles). If the track has a~monotonous tangent angle
			\begin{equation}
				\alpha(\tau) = \atan{\frac{\dv{X_2}{\tau}}{\dv{X_1}{\tau}}}
			\end{equation}
		with minimal and maximal $\alpha$ differing by less than~$\pi$ (i.e., the track changes direction by less than $180^\circ$), then all intersections of the track's normals must lie in an area bordered by the evolute and the normals at the beginning and the end of the curve (from their intersection with the evolute to their mutual intersection, see \cref{fig:rkdemo2,fig:rkdemo3}). Together, these three boundaries define a closed shape that will lie outside of the \ac{OFTPC} for a~typical track in our detector\footnote{The smallest anticipated radius of curvature is \qty{39}{\cm} for an electron or positron with a~kinetic energy \qty{3}{\MeV} in a~\qty{0.3}{\tesla} magnetic field. All points in the exclusion area must be farther from the track and therefore outside the \ac{OFTPC}.}.
			\begin{figure}
				\centering
				\includegraphics[width=0.55\textwidth]{rk_dist_demo.png}
				\hfill
				\includegraphics[width=0.42\textwidth]{rk_dist_demo2.png}
				\caption{An example track (red) with a~polygonal chain approximation (green, representing a~\ac{RK4} simulation). The distance of the point~$\mathbf{P}$ from the chain is found using a~binary search among the distances to the vertices $d_\mathbf{P}(\tau_i)$ (blue) and subsequently calculating the distance to segments neighboring the found vertex (thus finding the minimum of the function $d_\mathbf{P}'(\tau)$, function $d_\mathbf{P}(\tau)$ for the actual track is showed for reference). This approach works if the condition~\ref{eq:rk_assum} is satisfied, which is not the case for a~point from the green area bordered by the normals at endpoints and the evolute of the track (orange).}
				\label{fig:rkdemo2}
			\end{figure}
			\begin{figure}
				\centering
				\includegraphics[width=0.6\textwidth]{rk_dist_demo3.png}
				\caption{An exclusion area (green) of a~track (red) bordered by its evolute and the normals at endpoints (orange), where the assumption~\ref{eq:rk_assum} is violated. Unlike the track in \cref{fig:rkdemo2}, this track has a~minimal curvature point in the middle, corresponding to the cusp on its evolute.}
				\label{fig:rkdemo3}
			\end{figure}
		
		With the assumption~\ref{eq:rk_assum}, we can find the segment on the \ac{RK4} track with the lowest distance to a~given point~$\mathbf{P}$ using a~binary search algorithm. Let the distance of the point from the $n$\nobreakdash-th vertex %(resp. segment) 
		be~$d_{\mathbf{P},n} = d_{\mathbf{P}}(\tau_n)$%(resp.~$d_{\mathbf{P},n}'$). For every~$n$, there is a~$\tau_n'\in[\tau_{n-1},\tau_n]$ such that $d_{\mathbf{P}}(\tau_n') = d_{\mathbf{P},n}'$. Since~$d_{\mathbf{P}}(\tau)$ is a~continuous convex function and $\tau \in [0,\tau_N]$ for a~\ac{RK4} track with $N+1$~points, there is a~single minimum $d_{\textbf{P},\text{min}} = d_{\mathbf{P}}(\tau_\text{min})$ for some~$\tau_\text{min}\in[0,\tau_N]$%\in\{\tau_n'\}_{n=1}^N$.\footnote{the distance to two neighboring segments can be the same if the closest point is in their common vertex $\tau_n' = \tau_{n+1}'$}
		. Then the difference $\Delta d_{\mathbf{P},n} = d_{\mathbf{P},n}-d_{\mathbf{P},n-1}$ satisfies
			%\begin{equation}
			%	\begin{aligned}
			%		\Delta d_{\mathbf{P},n} &\leq 0\quad \forall n \in \{1,\ldots,k\},\\
			%		\Delta d_{\mathbf{P},n} &\geq 0\quad \forall n \in \{k+1,\ldots,N\}.
			%	\end{aligned}
			%\end{equation}
			\begin{equation}
				\begin{aligned}
					\Delta d_{\mathbf{P},n} &< 0\quad \forall n \text{ such that } \tau_n < \tau_\text{min},\\
					\Delta d_{\mathbf{P},n} &> 0\quad \forall n \text{ such that } \tau_{n-1} > \tau_\text{min}.
				\end{aligned}
			\end{equation}
		Therefore, we can search for the segment containing $d_{\textbf{P},\text{min}} = d_{\mathbf{P}}(\tau_\text{min})$ with binary search starting with $\Delta d_{\mathbf{P},1}$ and $\Delta d_{\mathbf{P},N}$, then calculate the difference $\Delta d_{\mathbf{P},m}$ for the middle index $m = \left\lfloor\frac{N+1}{2}\right\rfloor$. If $\Delta d_{\mathbf{P},m} > 0$\red{~(minor bug in the implementation -- if the value for the maximal index is negative, it shouldn't change anything)}, we can replace the higher index with~$m$, otherwise we replace the lower index. The~search stops when the difference between the minimal and maximal index is one.\red{~Would it be better if they were the same (maybe not)? Then the minimal value is $d_{\mathbf{P},n-1}$ or $d_{\mathbf{P},N}$ and we can take the minimum of the distances from the two segments connected to $n-1$. Currently taking the maximal index (and starting at $N-2$ maximal index $\leftrightarrow$ $N-1$\nobreakdash-th point), this should be equivalent, since either $\Delta d_{\mathbf{P},\text{max}} > 0$ (in the code is equivalent to max-1 here) or we are at $N-1$. The minimum of the two distances still taken.}
		
		\red{Same details with MIGRAD etc. as previously.}
		
		\subsection{Testing on a~microscopic sample}
			The Runge-Kutta fit together with the 3D circle-and-lines pre-fit was tested on a~sample of tracks simulated using the microscopic simulation described in Section~\ref{sec:microsim}. At first, few tracks with randomized initial parameters (same as the Runge-Kutta sample in Section~\ref{sec:rktest}) were generated for preliminary testing. Later, a~sample with a~grid-like distribution of track parameters was generated (see Section~\ref{sec:microgrid} for details).
			
			\red{Initial parameters of the HEED track (also should be in the first testing track). Initial parameters set in the circle fit (if electron set alpha one way, otherwise other way) and parameter bounds.}