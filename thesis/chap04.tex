\chapter{Energy Reconstruction}
\label{sec:energy}
	The~second stage of our reconstruction algorithm is the~reconstruction of the~particle's energy using its reconstructed track (see Section~\ref{sec:track}). We can achieve this by fitting the~track and extracting the~needed parameters of the~trajectory. We have tested three ways of reconstructing the~energy. Fitting is done using the~MINUIT algorithm implemented in ROOT~\cite{ROOT}. \textcolor{red}{Maybe cite some CERN article directly on MINUIT?}
	
	The~\textbf{Cubic Spline Fit} is a~rejected attempt at the~reconstruction of energy. It uses smoothly connected piecewise cubic polynomials between uniformly spaced nodes. Energy can then be computed using the~fit parameters by computing the~radius of curvature in different points of the~fitted curve using the~known magnitude of the~magnetic field perpendicular to the~trajectory. This approach was rejected because tuning the~fit to have a~reasonably stable radius of curvature is unpractical.
	
	The~\textbf{Circle and Lines Fit} was chosen as an~alternative since this corresponds to the~shape of a~trajectory of a~charged particle crossing a~finite volume with a~homogeneous magnetic field. The~energy of the~particle can be estimated using the~fitted radius and the~magnitude of the~perpendicular magnetic field in the~middle of the~\ac{TPC}.
	
	The~\textbf{Runge-Kutta Fit} uses the~4th order Runge-Kutta numerical integration described in Section~\ref{sec:rks}. Initial parameters of the~track (including the~particle's energy) are optimized so that the~integrated trajectory fits to the~reconstructed one. This fit can also be performed as a~single parameter (i.e., energy) fit if we can get the~initial position and orientation of the~particle on the~entrance to the~\ac{TPC} from previous detectors (\ac{Tpx3} and \ac{MWPC}, see Section~\ref{sec:IEAP}).
	
	\begin{figure}
		\centering
		\includegraphics[width=0.5\textwidth]{9010_3d.png}
		\caption{Example of a~fitted reconstructed track. \textcolor{red}{Swap for better image.}}
		\label{fig:90103d}
	\end{figure}
	
	\section{Cubic Spline Fit}
	\label{sec:cspline}
		The~first attempt to get an~early estimate of the~kinetic energy of the~particle uses a~cubic spline fit. This approach was later rejected in favor of the~circle and lines fit described in Section~\ref{sec:clines}. We use an~electron track starting in the~origin of our coordinate system with an~initial direction in the~positive $x$~axis. The~track is simulated microscopically (see Section~\ref{sec:microsim}) with a~kinetic energy of 8~MeV in a~gas mixture 90\%~Ar~+~10\%~CO$_2$ (the~same track was used in Section~\ref{sec:trackfirst}).
				
		In order to calculate the~spline, we use the~class \textit{TSpline3} from ROOT. This allows us to evaluate the~spline using the~coordinates $(x_n,z_n)$ of each node and the~derivatives $d_1,d_2$ in the~first and the~last node. We can fit these parameters of a~fixed amount of nodes to the~simulated trajectory. We use the~IMPROVE algorithm provided by the~\textit{TMinuit} class in ROOT. This algorithm attempts to find a~better local minimum after converging.
		
		After the~fit, we want to get an~energy estimate. We can calculate it at every point using the~radius of curvature of the~fitted spline. In ROOT, the~part of the~spline corresponding to a~given node is defined as
			\begin{equation}
				z(x) = z_n + b \Delta x+c(\Delta x)^2+d(\Delta x)^3,
			\end{equation}
		where $\Delta x = x-x_n$ and $b,c,d$ are coefficients. Using this equation, we can derive the~radius of curvature:
			\begin{equation}
				r(x) = \frac{\left(1+z'^2(x)\right)^\frac{3}{2}}{z''(x)} = \frac{\left(1+\left(b+2c\Delta x+3d(\Delta x)^2\right)^2\right)^\frac{3}{2}}{2c+6d\Delta x}.
			\end{equation}
		Based on the~geometry of the~detector, we can assume the~magnetic field \linebreak$\bm{B}(x,0,z) = (0,B(x,z),0)$ for a~track in the~XZ~plane. Since the~electron is relativistic, the~effect of the~electric field on its trajectory is negligible. The~Lorentz force $F_L$ is then always perpendicular to the~momentum of the~electron and is therefore equal to the~centripetal force $F_c$:
			\begin{align}
				F_L &= F_c,\\
				e\bm{v}\times\bm{B} &= \frac{\gamma m_e v^2}{r},\\
				e c\beta B &= \frac{E_{0e} \beta^2}{r\sqrt{1-\beta^2}},\\
				\sqrt{1-\beta^2} &= \frac{E_{0e} \beta}{ecBr},\\
				\beta^2(x) &= \frac{1}{1+\left(\frac{E_{0e}}{ecB(x,z(x))r(x)}\right)^2} \label{eq:ekin1}
			\end{align}
		where $e$~is the~elementary charge, $c$~is the~speed of light in vacuum, $m_e$~is the~rest mass of electron, $E_{0e} = m_e c^2$ is the~corresponding energy, $\gamma$~is the~Lorentz factor, $\bm{v}$~is the~velocity of the~electron, and $\beta = \frac{v}{c}$. We can then finally get our estimate of the~kinetic energy for a~given point on the~trajectory as follows:
			\begin{equation}
				\label{eq:ekin2}
				E_\text{kin}(x) = \left(\frac{1}{\sqrt{1-\beta^2(x)}}-1\right)E_{0e}.
			\end{equation}
		We can then average these estimates at multiple points to get one final estimate.
		\textcolor{red}{Add some figures.}
		
		\begin{figure}
			\centering
			\includegraphics[width=0.8\textwidth]{9010_splines.png}
			\caption{First attempt at a~track reconstruction using only the~drift velocity. Spline energy reconstruction attempt. \textcolor{red}{Swap for better image(s) -- subfigure environment., correct coordinates.}}
			\label{fig:9010splines}
		\end{figure}
	
	\section{Circle and Lines Fit}
	\label{sec:clines}
		A~simpler alternative for the~first estimation of the~particle's kinetic energy is a~fit of the~trajectory with a~circular arc with lines attached smoothly. This shape of trajectory corresponds to a~movement of a~charged particle through a~homogeneous magnetic field perpendicular to the~particle's momentum and limited to a~certain volume. In general, the~shape of such a~trajectory in a~non-perpendicularly oriented field is a~spiral. In our case, this component should be negligible since the~field is approximately toroidal and the~particle motion is nearly perpendicular to it. At first, we tested a~2D version of this fit, then we adapted it to 3D.
		
		Since our field is not homogeneous, it is not entirely clear what value of magnetic field should be used along with the~fitted radius (using equations~\ref{eq:ekin1} and~\ref{eq:ekin2}) to get the~best estimate for the~kinetic energy. Since we only use this method to get a~first rough estimate that we later refine, an~optimal solution of this problem is not required. Instead, we tested two options: taking the~value of the~field in the~middle of the~fitted circular arc and taking the~average field along it. \textcolor{red}{We haven't really tried to plot this for multiple tracks, but these estimates are saved somewhere and could be plotted.}
		
		\subsection{Two-dimensional fit}
			In the~2D case, the~fitted function used for the~electron track (which bends down, so we need to use the~upper part of the~circle) described in Section~\ref{sec:cspline} looks like this: \textcolor{red}{Maybe describe this track that we used at the~beginning somewhere earlier (section microscopic simulations \textrightarrow~Testing track?) so that it is easier to refer to it in multiple sections. It is not part of the~early GitHub commits, so maybe it won't be possible to create exact replicas of the~images, but they should be at least very similar.}
				\begin{equation}
					\label{eq:clines2d}
					z(x) = \begin{cases}
								a_1x+b_1 & x<x_1\\
								z_0+\sqrt{r^2-(x-x_0)^2} & x_1\leq x\leq x_2\\
								a_2x+b_2 & x>x_2
						   \end{cases},
				\end{equation}
			where $a_{1,2}$ and $b_{1,2}$ are the~parameters of the~lines, $(x_0,z_0)$ is the~center of the~circle, $r$ is its radius, and $(x_{1,2},z_{1,2})$ are the~coordinates of the~function's nodes. That means we have 9~parameters ($z_{1,2}$ is not used in the~function) along with 2~continuity conditions and 2~smoothness conditions. For the~fit, we use the~coordinates of the~nodes and the~radius of the~circle, which gives us 5~independent parameters (only the~radius has to be larger than half of the~distance between nodes). The~continuity conditions (combined with the~relations for $z_{1,2}$) are as follows:
				\begin{equation}
					\label{eq:ccont}
					z_{1,2} = a_{1,2}x_{1,2}+b_{1,2} = z_0-\sqrt{r^2-(x_{1,2}-x_0)^2}.
				\end{equation}
			The~smoothness conditions are as follows:
				\begin{equation}
					\label{eq:a12}
					a_{1,2} = \frac{x_0-x_{1,2}}{\sqrt{r^2-(x_{1,2}-x_0)^2}}.
				\end{equation}
			Equation~\ref{eq:ccont} gives us the~values of $b_{1,2}$
				\begin{equation}
					\label{eq:b12}
					b_{1,2} = z_{1,2} - a_{1,2} x_{1,2}.
				\end{equation}
			For the~coordinates of the~center of the~circle, we can use the~fact that the~center has to lie on the~axis of its chord. In other words, there is a~value of a~parameter~$t$ such that, using the~parametric equation of the~axis
				\begin{equation}
					\begin{pmatrix} x_0\\ z_0 \end{pmatrix} = \begin{pmatrix} \frac{x_1+x_2}{2}\\ \frac{z_1+z_2}{2} \end{pmatrix} + t \begin{pmatrix} \frac{z_2-z_1}{2}\\ \frac{x_1-x_2}{2} \end{pmatrix}.
				\end{equation}
			At the~same time, the~center has to be in a~distance of $r$ from the~nodes:
				\begin{gather}
					(x_1-x_0)^2 + (z_1-z_0)^2 = r^2,\\
					\left(\frac{x_1-x_2}{2}+\frac{z_1-z_2}{2}t\right)^2 + \left(\frac{z_1-z_2}{2}+\frac{x_2-x_1}{2}t\right)^2 = r^2,\\
					\left(\left(\frac{x_2-x_1}{2}\right)^2+\left(\frac{z_2-z_1}{2}\right)^2\right)t^2+\left(\frac{x_2-x_1}{2}\right)^2+\left(\frac{z_2-z_1}{2}\right)^2-r^2=0.
				\end{gather}
			Since our electron track bends towards negative $z$ and $x_2 > x_1$, we only care about the~solution with $t>0$
				\begin{equation}
					t = \sqrt{\frac{r^2}{\left(\frac{x_2-x_1}{2}\right)^2+\left(\frac{z_2-z_1}{2}\right)^2}-1},
				\end{equation}
				\begin{align}
					x_0 = \frac{x_1+x_2}{2} + \frac{z_2-z_1}{2} \sqrt{\frac{r^2}{\left(\frac{x_2-x_1}{2}\right)^2+\left(\frac{z_2-z_1}{2}\right)^2}-1},\label{eq:x0}\\
					z_0 = \frac{z_1+z_2}{2} - \frac{x_2-x_1}{2} \sqrt{\frac{r^2}{\left(\frac{x_2-x_1}{2}\right)^2+\left(\frac{z_2-z_1}{2}\right)^2}-1}.\label{eq:z0}
				\end{align}
			The~function defined in the~equation~\ref{eq:clines2d} along with equations~\ref{eq:a12}, \ref{eq:b12}, \ref{eq:x0} and~\ref{eq:z0} derived using the~continuity and smoothness conditions (combined with the~relations for $z_{1,2}$) fully define our fitted function with parameters $r,x_{1,2},z_{1,2}$. \textcolor{red}{Some pictures of the~fit on the~tested track. Results of the~fit. Again, the~actual fit uses 8-z. Use GeoGebra schematics to generate a~picture of 2D geometry.}
			
			\textcolor{red}{Energy reconstruction with circle and lines fit. Trilinear interpolation of the~magnetic field. Tested on a~Runge-Kutta sample; future testing with microscopic simulations and map simulation. Preliminary 2D version (done) and complete 3D version. Geometry of the~fit with its derivation.}
			
			\begin{figure}
				\centering
				\includegraphics[width=0.8\textwidth]{9010_circle2D.png}
				\caption{First attempt at a~track reconstruction using only the~drift velocity. Circle and Lines Fit in 2D. \textcolor{red}{Swap for better image, correct coordinates.}}
				\label{fig:9010circle2D}
			\end{figure}
		
		\subsection{Three-dimensional fit}
			\textcolor{red}{Explain the geometry and least square method used for the~3D fit.}
			
			\begin{figure}
				\centering
				\includegraphics[width=0.8\textwidth]{circlefit.png}
				\caption{Circle and Lines Fit 3D geometry. \textcolor{red}{Swap for better image.}}
				\label{fig:circlefit}
			\end{figure}
	
	\section{Runge-Kutta Fit}
		\textcolor{red}{Single parameter fit with 4th order Runge-Kutta simulated track. Future testing with microscopic simulations and map simulation. Derivation of the~geometry (least squares).}