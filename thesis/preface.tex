\chapwithtoc{Motivation}
	A~\ac{TPC} is a~type of gaseous detector that detects charged particle trajectories by measuring the~positions and drift time of ions created in the~gas; details are provided in Section~\ref{sec:tpc}. The~energy of these particles can be inferred from the~curvature of their trajectory in the~magnetic field.
	
	The~goal of this thesis is to develop an~algorithm for the~reconstruction of charged particle trajectories and energy in an~atypic \ac{TPC} with orthogonal electric and magnetic fields, hereafter referred to as the \ac{OFTPC}, used in the~X17 project at the~\ac{IEAPCTU}. Furthermore, we present the~results of testing this algorithm with different samples of simulated data. We use the~\garfieldpp toolkit~\cite{Garfield++} for simulations in combination with the~ROOT~framework~\cite{ROOT} for data analysis and visualization. Some of our more demanding simulations are run on the~MetaCentrum grid.
	
	The~X17 project in \ac{IEAPCTU} aims to reproduce measurements of anomalous behavior in the~angular correlation distribution of pairs produced by the~\ac{IPF} mechanism during the~decay of certain excited nuclei (\iso{Be}{8},~\iso{C}{12},~and~\iso{He}{4}) observed by the~ATOMKI group in Hungary. \textcolor{red}{I would leave this here as a short summary before I explain it in more detail in the sections below.}
	
	\textcolor{red}{Add citations: MetaCentrum, X17 project, VdG, ATOMKI papers. Maybe also TPC, IPF, etc.}
	
	\section{ATOMKI Measurements}
	\textcolor{red}{Short summary of results of measurements in ATOMKI.}
	
	\section{X17 Project at IEAP CTU}
	\label{sec:IEAP}
		\textcolor{red}{Short summary of our goals, maybe mention the~grant.}