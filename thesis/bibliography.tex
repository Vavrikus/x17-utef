%%% Bibliography (literature used as a source)
%%%
%%% We employ biblatex to construct the bibliography. It processes
%%% citations in the text (e.g., the \cite{...} macro) and looks up
%%% relevant entries in the bibliography.bib file.
%%%
%%% See also biblatex settings in thesis.tex.

%%% Generate the bibliography. Beware that if you cited no works,
%%% the empty list will be omitted completely.

% We let bibliography items stick out of the right margin a little
\def\bibfont{\hfuzz=2pt}

\printbibliography[heading=bibintoc]
\vspace{1em}
\noindent \textbf{Acknowledgments:}
\begin{itemize}[nosep]
	\item Computational resources were provided by the e-INFRA CZ project (ID:90254), supported by the Ministry of Education, Youth and Sports of the Czech Republic.
	\item \Cref{fig:oftpc,fig:trilin,fig:microgrid,fig:cfit3d,fig:rkdemo,fig:rkdemo2,fig:rkdemo3} were made with GeoGebra®.
	\red{\item Grant?}
\end{itemize}


%%% In case you prefer to write the bibliography manually (without biblatex),
%%% you can use the following. Please follow the ISO 690 standard and
%%% citation conventions of your field of research.

% \begin{thebibliography}{99}
	%
	% \bibitem{lamport94}
	%   {\sc Lamport,} Leslie.
	%   \emph{\LaTeX: A Document Preparation System}.
	%   2nd edition.
	%   Massachusetts: Addison Wesley, 1994.
	%   ISBN 0-201-52983-1.
	%
	% \end{thebibliography}