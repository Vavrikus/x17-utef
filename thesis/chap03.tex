\chapter{Track Reconstruction}
\label{sec:track}
	As the first step of the reconstruction algorithm, we reconstruct the track of a~primary particle -- either an electron or a~positron. Then, using this information, we determine the energy of the particle (\cref{sec:energy}).
	
	The \textbf{\acf{RASD}} uses the standard \ac{TPC} approach. With parallel fields, the drift inside a~uniform electric field remains undistorted (as shown in \cref{eq:drift}). Therefore, we only need to reconstruct the $z$\nobreakdash-coordinate from the drift time using the known drift velocity. We also assume that the readout coordinates ($x'$,~$y'$,~$t$) are known exactly, neglecting the pads and time binning.
	
	Reconstruction using an \textbf{Ionization Electron Map} (from now on referred to as \emph{the map}) uses a~simulation of the drift of secondary (ionization) electrons within the detector volume. This simulation can then be used to interpolate the initial position of the secondary electrons. In the first iteration of this method the readout is assumed to be continuous.
	
	We present two algorithms using the map for reconstruction. The first one uses a~gradient descent algorithm along with trilinear interpolation (see \cref{sec:trilin}) of the map. The second method uses interpolation on the irregular inverse grid with a~polynomial.
	
	The \textbf{Discrete Reconstruction} uses the map; instead of reconstructing the exact position of each electron, we reconstruct the center of each hit pad together with the time corresponding to the midpoint of the time bin. The electron count in each \ac{OFTPC}~bin (consisting of the pad and the time bin) serves as an idealized collected charge, which is then used as a~weight in the energy reconstruction fit.
	
	\section{Reconstruction Assuming Steady Drift}
	\label{sec:rasd}
		As the first step, we tried to reconstruct a~simulated electron track with a~special set of initial parameters, described in detail in \cref{sec:microfirst}. The starting point is given by the origin of our coordinate system and its initial direction is given by the positive $x$\nobreakdash-axis. This means the magnetic field of our detector is perpendicular to the momentum of the particle at all times, and we can reduce the problem to two-dimensional space.
		
		For the reconstruction, we decided to use the common method used in a~standard \ac{TPC}\cite{TPCs}. We call this method \textit{\acf{RASD}}. This will allow us to explore the significance of the atypical behavior in our \ac{OFTPC}. Additionally, we assume the readout is continuous to further simplify the problem. In this approximation, we reconstruct the initial position of each ionization electron.
		
		The reconstruction is then defined by the following relations between the coordinates of the detector space and the readout space (see \cref{sec:coor}):
			\begin{align}
				x &= x',\\
				y &= y',\\
				z &= \qty{8}{\cm} - d_\text{r} = \qty{8}{\cm} - v_\text{d} t,
			\end{align}
		where $d_\text{r}$ is the distance to the readout, and $v_\text{d}$ is the drift velocity of electrons in the given gas mixture. At a~phenomenological level, this velocity can be considered as a~function of the electric field~$\mathbf{E}$ and the magnetic field~$\mathbf{B}$ as shown in \cref{eq:drift}. Since we assume a~uniform electric field in the detector and in this~approximation we want to neglect the effect of our unusual magnetic field, we consider the drift velocity constant. We can estimate the drift velocity by fitting the dependence $d_\text{r}(t)$ of ionization electrons from a~simulated track with a~linear function:
			\begin{equation}
				d_\text{r}(t) = v_\text{d} t + d_0.
			\end{equation}
		The fit was applied on two tracks with different gas composition, the result is in \cref{fig:zt}. The obtained parameters are then used for the reconstruction shown in \cref{fig:rasd_xz}. From the residuals shown in \cref{fig:rasd_res}, we can see that this reconstruction algorithm leads to significant deviations from the simulated track (up to 1.1~cm for 90:10, and up to 0.3~cm for 70:30 Ar:CO$_2$), especially in the faster gas mixture 90:10 (as expected -- for a~higher mean time between collisions in \cref{eq:drift}, the effect of the magnetic field is bigger). These deviations are mainly caused by the shift in the $x$\nobreakdash-coordinate due to the tilt of the drift lines in magnetic field. In order to account for this, we need to develop a~better algorithm.
		
		\begin{figure}
			\centering
			\includegraphics[width=0.48\textwidth]{9010_zt.png}
			\hfill
			\includegraphics[width=0.48\textwidth]{7030_zt.png}
			\caption{Linear fit of the drift time $t$ dependence on the distance to the readout $d_\text{r} = \qty{8}{cm} - z$ for the ionization electrons in 90:10 (left) and 70:30 (right) Ar:CO$2$ gas composition. Only electrons inside the \ac{OFTPC} (red) are fitted. The parameters are $v_\text{d} = \qty{3.39}{\cm\per\us}$, $d_0 = -\qty{0.41}{cm}$ for 90:10, and $v_d = \qty{0.939}{\cm\per\us}$, $d_0 = -\qty{0.11}{cm}$ for 70:30 Ar:CO$_2$.}
			\label{fig:zt}
		\end{figure}
		
		\begin{figure}
			\centering
			\includegraphics[width=0.48\textwidth]{9010_xz_rasd.png}
			\hfill
			\includegraphics[width=0.48\textwidth]{7030_xz_rasd.png}
			\caption{Reconstruction (black) of the starting position of ionization electrons (red) using parameters obtained from the fit (\cref{fig:zt}). Two gas compositions 90:10 (left) and 70:30 Ar:CO$_2$ (right) are compared.}
			\label{fig:rasd_xz}
		\end{figure}
		
		\begin{figure}
			\centering
			\includegraphics[width=0.7\textwidth]{rasd_res.png}
			\caption{Comparison of residuals (i.e., the distance from the reconstructed point to the simulated ionization electron starting point) dependence on~$x$ for two gas mixtures 90:10 (red) and 70:30 Ar:CO$_2$ (blue).}
			\label{fig:rasd_res}
		\end{figure}
	
	\section{Ionization Electron Map}
	\label{sec:map}
		Inside an \ac{OFTPC}, the drift of the ionization electrons is significantly affected by its magnetic field as shown in \cref{eq:drift}, see also \cref{fig:microfirst}. We need to take this into account for accurate reconstruction. In the first approximation, we assume a~continuous readout (i.e., we neglect the anode segmentation into pads). We can then reconstruct the original position of each ionization electron using its readout coordinates. For this purpose, we use the ionization electron map.
		
		The ionization electron map represents a~mapping from the detector space to the readout space (see \cref{sec:coor}). It tells us what readout coordinates $(x',y',t)$ we can expect on average for an ionization electron created at the detector coordinates $(x,y,z)$. More precisely, it is a~mapping to the distributions on the readout space; we can simplify this as only the means $\overline{\mathcal{M}}$ and the covariance matrices $\mathcal{M}_\mathbf{\Sigma}$, assuming Gaussian distribution\footnote{The applicability of Gaussian distribution was tested on simulated points of the map using the Mardia's test.}:
			\begin{alignat}{3}
				\overline{\mathcal{M}}&\colon \mathcal{D} \longrightarrow \mathcal{R}, &&(x,y,z) \longmapsto \mathbf{\overline{X}}^T\equiv(\bar{x}',\bar{y}',\bar{t}),\\
				\mathcal{M}_\mathbf{\Sigma}&\colon \mathcal{D} \longrightarrow \mathbb{R}^{3\times3}, &&(x,y,z) \longmapsto \mathbf{\Sigma}^{\phantom{T}} \equiv \scalebox{0.9}{$\begin{pmatrix}
					\sigma_{x'}^2 & \operatorname{cov}(x',y') & \operatorname{cov}(x',t) \\
					\operatorname{cov}(y',x') & \sigma_{y'}^2 & \operatorname{cov}(y',t) \\
					\operatorname{cov}(t,x') & \operatorname{cov}(t,y') & \sigma_{t}^2
				\end{pmatrix}$},\\
				\mathcal{M}&\colon \mathcal{D} \longrightarrow D(\mathcal{R}),\ &&(x,y,z) \longmapsto N(\mathbf{X}) \equiv \frac{\exp(-\frac{1}{2}(\mathbf{X}-\mathbf{\overline{X}})^T\mathbf{\Sigma}(\mathbf{X}-\mathbf{\overline{X}}))}{\sqrt{(2\pi)^3 |\mathbf{\Sigma}|}}.
			\end{alignat}
		To get an approximation of this mapping, we simulate the drift of ionization electrons generated on a~regular Cartesian grid $\mathbb{G}\subset\mathcal{D}$ with spacing~$l$ inside the volume of our \ac{OFTPC}\footnote{The detector walls are not considered and we simulate the drift even outside of the \ac{OFTPC} which allows us to interpolate even close to the walls.} (see the visualization in \cref{fig:map_3d}). In \cref{fig:maplines}, you can see an example of drift lines from a~test of the simulation. After testing runs, two map simulations were made with different gas composition, their parameters are shown in \cref{tab:map}. 
		
		\begin{figure}
			\centering
			\includegraphics[width=\textwidth]{map_3dvis.png}
			\caption{A~3D visualization of the mapping of means $\overline{\mathcal{M}}$ for the 90:10 Ar:CO$_2$ gas. A regular grid $\mathbb{G}$ with $l = \qty{1}{\centi\meter}$ in the detector space is mapped to an irregular grid $\mathbb{G}^{-1} \equiv \overline{\mathcal{M}}(\mathbb{G})$ in the readout space. Made with GeoGebra®.}
			\label{fig:map_3d}
		\end{figure}
		
		\begin{figure}
			\centering
			\includegraphics[width=0.5\textwidth]{map_lines.png}
			\caption{A test of the 90:10 Ar:CO$_2$ map simulation with spacing $l = \qty{1.5}{cm}$. The resulting drift lines of evenly spaced electrons are displayed in orange.}
			\label{fig:maplines}
		\end{figure}
		
		\begin{table}
			\centering
			\caption{Comparison of parameters of two map simulations.}
			{\renewcommand{\arraystretch}{1.2}
			\begin{tabular}{|c|c|c|}
				\hline
				\textbf{Parameter} & \textbf{90:10 Ar:CO$_2$ map} & \textbf{70:30 Ar:CO$_2$ map}\\
				\hline
				$N$ & 100 & 100 \\
				\hline
				$l$ & \qty{1.0}{\centi\meter} & \qty{0.5}{\centi\meter} \\
				\hline
				$z$ bounds & \qtyrange{-8}{8}{\centi\meter} & \qtyrange{-8}{8}{\centi\meter} \\
				\hline
				$x$ bounds & \qtyrange{0}{15}{\centi\meter} & \qtyrange{-1.5}{15.0}{\centi\meter} \\
				\hline
				$y$ bounds & $|y| \leq x\cdot\tan\frac{\pi}{3}$ & $|y| \leq \left(x+\qty{1.5}{\centi\meter}\right)\cdot\tan\frac{\pi}{6}$ \\
				\hline
				initial energy & \qty{0.1}{\eV} & \qty{0.1}{\eV} \\
				\hline
				init. direction & randomized & randomized \\
				\hline
			\end{tabular}}
			\label{tab:map}
		\end{table}
		
		In order to get accurate results, we use the microscopic simulation of these electrons described in \cref{sec:microsim}. It is also useful to simulate multiple~($N$) electrons originating from the same position so that we can account for the random fluctuations due to collisions. Using the readout coordinates of the electrons, we then estimate the means and the covariance matrix:
			\begin{equation}
				\label{eq:cov}
				\mathbf{\overline{X}} = \frac{1}{N}\sum_{i=1}^{N} \mathbf{X}_i,\qquad \mathbf{\Sigma} = \frac{1}{N-1}\sum_{i=1}^{N}(\mathbf{X}_i-\mathbf{\overline{X}})(\mathbf{X}_i-\mathbf{\overline{X}})^T,
			\end{equation}
		where $\mathbf{X}_i$ represents the readout coordinates $(x'_i, y'_i, t_i)^T$ of the $i$\nobreakdash-th electron. The matrix (resp. its submatrix) can then be used to plot error ellipsoid (resp. ellipse). The axes correspond to the eigenvectors, errors along these axes for a~given confidence level~$p$ can be computed using the chi\nobreakdash-squared distribution
			\begin{equation}
				\label{eq:sigma}
				\sigma_i = \sqrt{\lambda_i \chi^2_k(p)},
			\end{equation}
		where $\lambda_i$ is the corresponding eigenvalue and $k$~is the number of degrees of freedom.
		
		As shown in \cref{fig:map_zt,fig:map_xt}, the drift times in the map are no longer proportional to the $z$\nobreakdash-coordinate due to the varying Lorentz angles in the inhomogeneous magnetic field (see \cref{eq:lorentz}). As expected, the effect is considerably larger in gases with higher drift velocities. Similarly, the drift distortion (i.e., its deviation from the vertical lines) is huge for the "faster" gas, but still significant for the "slower" one, as demonstrated in \cref{fig:map_yx_all,fig:map_bot,fig:map_xz}.
		
		\begin{figure}
			\centering
			\includegraphics[width=0.48\textwidth]{map_zt_9010.png}
			\hfill
			\includegraphics[width=0.48\textwidth]{map_zt_7030.png}
			\caption{Dependence of the drift times of the simulated map $\overline{\mathcal{M}}$ on the $z$\protect\nobreakdash-coordinate. Two gas mixtures 90:10 (left) and 70:30 Ar:CO$_2$ (right) are compared. The spread is caused by varying Lorentz angles.}
			\label{fig:map_zt}
		\end{figure}
		
		\begin{figure}
			\centering
			\includegraphics[width=0.48\textwidth]{map_xt_9010.png}
			\hfill
			\includegraphics[width=0.48\textwidth]{map_xt_7030.png}
			\caption{The $x't$ projection of the $\mathcal{M}(\mathbb{G}_{y=0})$ mapping of a~part of the regular grid $\mathbb{G}$. The means $\overline{\mathcal{M}}(\mathbb{G}_{y=0})$ are marked with red crosses, and the diffusion error is denoted by black 95\% confidence error ellipses computed from the diagonalized covariance matrices $\mathcal{M}_{\mathbf{\Sigma}}(\mathbb{G}_{y=0})$. Two gas mixtures 90:10 (left) and 70:30 Ar:CO$_2$ (right) are compared. The first mixture shows differences of~$t$ for electrons with same initial~$z$ but different initial~$x$. For the second mixture, these differences are negligible in comparison with the diffusion.}
			\label{fig:map_xt}
		\end{figure}
		
		\begin{figure}
			\centering
			\includegraphics[width=0.48\textwidth]{map_yx_all_9010.png}
			\hfill
			\includegraphics[width=0.48\textwidth]{map_yx_all_7030.png}
			\caption{The regular grid $\mathbb{G}$ projected by the mapping $\overline{\mathcal{M}}$ from the detector space onto the $x'y'$~plane ($t$ is not plotted). Layers with lower $z$\protect\nobreakdash-coordinate (i.e., further away from the readout) are displayed with darker colors. The \ac{OFTPC} volume is marked with black lines. Two gas mixtures 90:10 (left) and 70:30 Ar:CO$_2$ (right) are compared.}
			\label{fig:map_yx_all}
		\end{figure}
		
		\begin{figure}
			\centering
			\begin{subfigure}[t]{\textwidth}
				\centering
				\includegraphics[width=0.48\textwidth]{map_yx_bottom_9010.png}
				\hfill
				\includegraphics[width=0.48\textwidth]{map_yx_bottom_7030.png}
				\caption{The $x'y'$ projection of $\mathcal{M}(\mathbb{G}_{-8})$ (similar as in \cref{fig:map_yx_all}), the diffusion is denoted with the 95\% error ellipses from  the diagonalized sample covariance matrices $\mathcal{M}_\mathbf{\Sigma}(\mathbb{G}_{-8})$ $\leftrightarrow$ \cref{eq:cov}, and computed using \cref{eq:sigma}. The mean values $\overline{\mathcal{M}}(\mathbb{G}_{-8})$ are connected by black arrows with the corresponding starting position $(x,y)$ of the simulated electrons. The \ac{OFTPC} volume is marked with black lines.}
				\label{fig:map_yx_bot}
			\end{subfigure}
			
			\begin{subfigure}[t]{\textwidth}
				\centering
				\includegraphics[width=0.48\textwidth]{map_xyt_9010.png}
				\hfill
				\includegraphics[width=0.48\textwidth]{map_xyt_7030.png}
				\caption{The full mapping $\mathcal{M}(\mathbb{G}_{-8})$, the diffusion is marked using standard error bars (black) from the diagonalized sample covariance matrices (\cref{eq:cov,eq:sigma}).}
				\label{fig:map_xyt}
			\end{subfigure}
			\caption{The $\mathcal{M}(\mathbb{G}_{-8})$ mapping of the bottom ($z = \qty{-8}{\centi\meter}$) layer $\mathbb{G}_{-8}$ of the regular grid $\mathbb{G}\subset\mathcal{D}$. It includes both the mapping of means $\overline{\mathcal{M}}$ and of covariances $\mathcal{M}_\mathbf{\Sigma}$. Individual electrons from the map simulation are marked with red dots. Two gas mixtures 90:10 (left) and 70:30 Ar:CO$_2$ (right) are compared.}
			\label{fig:map_bot}
		\end{figure}
		
		\begin{figure}
			\centering
			\includegraphics[width=0.48\textwidth]{map_xz_9010.png}
			\hfill
			\includegraphics[width=0.48\textwidth]{map_xz_7030.png}
			\caption{The readout coordinate~$x'$ for points on the grid $\mathbb{G}_{y=0}$ plotted against their initial coordinate $z$. The means are marked with red crosses, the diffusion in~$x'$ is denoted by standard error bars. Two gas mixtures 90:10 (left) and 70:30 Ar:CO$_2$ (right) are compared.}
			\label{fig:map_xz}
		\end{figure}
		
		When evaluating the map inside the grid, we use trilinear interpolation (see \cref{sec:trilin}). From now on, we will use the same symbol $\mathcal{M}$ for this interpolated simulation.
		
		Finally, we need to invert the map to get the original detector coordinates $(x,y,z)$ from the given readout coordinates $(x',y',t)$. In our case, it is reasonable to assume that the mapping $\overline{\mathcal{M}}$ (we lose the information about the distribution, although it might be possible to recover through complex techniques; an idea, how to do this, is posed at the end of the thesis, though likely unpractical). We implemented two methods for this purpose: the gradient descent search (\cref{sec:grad}) and interpolation on the inverse grid (\cref{sec:interpol}).
		
		\subsection{Map Simulation}
			Simulation of the map is a~computationally heavy task. For this reason, we use the MetaCentrum grid~\cite{metacentrum} to parallelize needed calculations. At first, this was done by evenly distributing the simulated electrons across the individual jobs in a~simulation with only one electron per vertex in the regular grid~$\mathbb{G}$ with a~spacing of one centimeter. Later, a~more efficient approach was implemented, accounting for the varying lengths of the drift of individual electrons. If we index the vertices of~$\mathbb{G}$ in the order of increasing coordinates $y,x,z$, we can express the number~$n_l$ of full XY~layers (i.e., electrons with the same $z$~coordinate, the mapping of one such layer is shown in \cref{fig:map_xyt}) with index less than or equal to $i$
				\begin{equation}
					n_l(i) = \left\lfloor\frac{i}{n_{xy}}\right\rfloor,
				\end{equation}
			where $n_{xy}$ is the number of electrons in each XY~layer calculated simply by counting the electrons that satisfy boundary conditions for $x$~and~$y$. The number of electrons remaining in the top layer is then
				\begin{equation}
					n_r(i) = i\!\!\!\!\mod n_{xy}.
				\end{equation}
			Finally, we can calculate the sum of the drift gaps of electrons up to index~$i$
				\begin{equation}
					d_\text{sum} = (z_\text{max}-z_\text{min})n_{xy}n_l-\frac{n_l(n_l-1)}{2}n_{xy}l+n_r(z_\text{max}-z_\text{min}-n_l l).
				\end{equation}
			We then use a~binary search algorithm to find the maximum index $i$ such that the value of this sum is less than the fraction $\frac{\text{job id}}{\text{max job id}}$ of the total sum. This way we obtain the minimal and the maximal index of electrons simulated in the given job. The obtained map is then stored in a~custom class template \texttt{Field<MapPoint>} (used also for the magnetic field --- \texttt{Field<Vector>}).
		
		\subsection{Gradient Descent Algorithm}
		\label{sec:grad}			
			The first implemented method of reconstruction uses a~gradient descent algorithm to calculate an inversion of the map $\overline{\mathcal{M}}$ in a~given point. Gradient descent is an iterative minimization algorithm for multivariate functions. Let $\mathbf{R}\in\mathcal{R}$ be a~point in the readout space; we want to find a~point $\mathbf{D} = (x,y,z) \in\mathcal{D}$ in the detector space such that 
				\begin{equation}
					\overline{\mathcal{M}}(\mathbf{D}) = \mathbf{R} = (x'_\mathbf{R},y'_\mathbf{R},t_\mathbf{R}).
				\end{equation}
			We define a~function~$f_\mathbf{R}$ in the readout space as a~distance in this space:
				\begin{equation}
					f_\mathbf{R}(x',y',t) = \sqrt{(x'-x'_\mathbf{R})^2+(y'-y'_\mathbf{R})^2+v_\text{d}^2(t-t_\mathbf{R})^2},
				\end{equation}
			where $v_d$ is an approximation of the drift velocity in the \ac{OFTPC}, obtained from the reconstruction in \cref{sec:rasd} (see \cref{fig:rasd_xz}). We make an initial guess:
				\begin{equation}
					\mathbf{D}_0 = (x'_\mathbf{R},y'_\mathbf{R},v_\text{d}t).
				\end{equation}
			Assuming we have the $n$-th estimate $\mathbf{D}_n$, we calculate the $i$-th component of the gradient of $f_\mathbf{R}\circ\overline{\mathcal{M}}$ numerically using central differences:
				\begin{equation}
					\left[\nabla(f_\mathbf{R}\circ\overline{\mathcal{M}})\right]^i(\mathbf{D}_n) \approx \frac{f_\mathbf{R}(\overline{\mathcal{M}}(\mathbf{D}_n+s\cdot e^i))-f_\mathbf{R}(\overline{\mathcal{M}}(\mathbf{D}_n-s\cdot e^i))}{2s},
				\end{equation}
			where $e^i\in\mathcal{D}$ is the $i$-th coordinate vector and $s$ is the step size. The step size should be sufficiently small; initially, we set it as a~fraction $s = \frac{l}{10}$ of the map's grid spacing $l$. During the minimization, we check that $f_R(\overline{\mathcal{M}}(\mathbf{D}_n))<10s$ at all times and decrease the step if needed. The next iteration can be calculated as follows:
				\begin{equation}
					\mathbf{D}_{n+1} = \mathbf{D}_n - \gamma \nabla(f_\mathbf{R}\circ\overline{\mathcal{M}})(\mathbf{D}_n),
				\end{equation}
			where $\gamma\in\mathbb{R}^+$ is the damping coefficient. It should be set to a~small enough value to ensure convergence, but large enough for sufficient converging speed. In order to avoid oscillation around the minimum, we also check that the size of the damped gradient does not exceed (or get too close to) the current value of the non-negative function:
				\begin{equation}
					10 \norm{\gamma \nabla(f_\mathbf{R}\circ\overline{\mathcal{M}})(\mathbf{D}_n)} < f_\mathbf{R}(\mathbf{D}_n),
				\end{equation}
			and lower $\gamma$ otherwise. The minimization stops either when the error $f_\mathbf{R}(\overline{\mathcal{M}}(\mathbf{D}_n))$ drops below a~specified value or when the number of iterations exceeds a~certain limit (in this case, a~message is printed into the console).
			
			The parameters of this method can be further optimized (e.g., a~better choice of $\gamma$ and~$s$, analytical computation of the gradient\footnote{The gradient can be computed analytically when using the trilinear interpolation, see \cref{eq:trilinpoly}.}); instead, we later decided to use the interpolation on the inverse grid described in the next section.
		
		\subsection{Interpolation on the Inverse Grid}
		\label{sec:interpol}			
			The best current algorithm uses the interpolation on the inverse grid. Rather than inverting the trilinearly interpolated map using a~numerical minimization method as in the previous section, we take advantage of the fact that the map $\overline{\mathcal{M}}$ is a~bijection in the area of interest. Since we have simulated values of this map on a~regular grid in the detector space $\mathcal{D}$, we also know the inverse map $\overline{\mathcal{M}}^{-1}$ on the irregular inverse grid in the readout space $\mathcal{R}$. To get an approximation of the inverse map in the entire readout space, we can use interpolation.
			
			Since the inverse grid is irregular, trilinear interpolation cannot be applied. Given that the simulated map is dense enough to provide a~good approximation considering the size of our pads, we can adopt a~similar approach.\footnote{A more complicated and computationally heavy alternative would be natural neighbor interpolation or Kriging.} As shown in \cref{eq:trilinpoly} in \cref{sec:trilin}, trilinear interpolation can be expressed as a~polynomial:
				\begin{equation}
					\widehat{f}(x,y,z) = axyz + bxy + cxz + dyz + ex + fy + gz + h,
				\end{equation}
			where $a,b,c,d,e,f,g,h$ are coefficients uniquely determined by the values of the function at the vertices of the interpolation cell. We can generalize this for a~function defined on an irregular grid. Given the function values at any eight points, we can write a~system of eight linear equations
				\begin{equation}
					\begin{pmatrix}
						x_1 y_1 z_1 & x_1 y_1 & x_1 z_1 & y_1 z_1 & x_1 & y_1 & z_1 & 1\\
						\vdots & \vdots & \vdots & \vdots & \vdots & \vdots & \vdots & \vdots\\
						x_8 y_8 z_8 & x_8 y_8 & x_8 z_8 & y_8 z_8 & x_8 & y_8 & z_8 & 1
					\end{pmatrix}
					\begin{pmatrix}
						a\\
						\vdots\\
						h
					\end{pmatrix}
					=
					\begin{pmatrix}
						f(x_1,y_1,z_1)\\
						\vdots\\
						f(x_8,y_8,z_8)
					\end{pmatrix},
				\end{equation}
			which has a~unique solution for the coefficients for most values of $(x_n, y_n, z_n)$ and $f(x_n,y_n,z_n)$, where $n\in\{1,\ldots,8\}$.
			
			This approach introduces a~small complication: finding the correct pseudocell (i.e., the image of eight vertices forming a~cubic cell in the regular grid) in the inverse grid. The eight irregularly spaced vertices of this pseudocell do not define a~unique volume, so there are multiple possible ways to partition $\mathcal{R}$ into pseudocells, with no obvious choice among them. This problem could be solved by determining the exact volume boundary using the gradient descent search described above. However, since the difference between the interpolation polynomials is small near the boundary of the pseudocell, we choose one of them arbitrarily.
			
			In the code, the map is stored in an array, and its individual points can be accessed using three indexes corresponding to the $x$, $y$, and~$z$ coordinates. Points outside of the area of the simulation are initialized with zeros. The search for the pseudocell finds the indices corresponding to the eight vertices. It starts by performing a~binary search along the $x$\nobreakdash-coordinate (comparing the $x'$\nobreakdash-coordinate of the map vertices with that of the point, whose preimage we want to find) with indices corresponding to~$y$ and~$z$ fixed to half of their maximal value. After that, we fix the $x$ and $z$ indices and perform a~binary search along~$y$ (making sure to exclude the zeros from initialization outside the simulation range beforehand), and finally, we fix the $x$ and $y$ indices and perform a~search along~$z$ (inverted, since higher~$z$ corresponds to lower~$t$). This procedure is then repeated until the pseudocell "contains" the preimage of the given point (meaning that for all readout coordinates $x'$, $y'$, $t$, there is a~vertex of the pseudocell with a~higher and a~vertex with a~lower value of that coordinate) or until 3 iterations are exceeded. We also attempt to fix the error by choosing a~neighboring cell.
			
			This approach should generally be faster than the gradient descent method, since we do not need to iterate. Currently, we are calculating the coefficients of the interpolation polynomial during the reconstruction of every point. If further optimization is needed in the future, we can precalculate them for the whole map.
			
		\section{Reconstruction testing}
			The continuous reconstruction using the map was tested on microscopic tracks with parameters described in \cref{sec:microfirst}. The difference between the reconstructed positions obtained using the two map inversion algorithms was found to be smaller than the uncertainty from diffusion by several orders of magnitude. In its unoptimized state, the gradient descent search was significantly slower (at least ten times) than the interpolation on the inverse grid.
			
			An example of reconstruction of tracks in 90:10 and 70:30 Ar:CO$_2$ gas mixture is shown in \cref{fig:reco_map}. We can clearly see that the results have improved compare to the \acf{RASD} (shown in \cref{fig:rasd_xz}). A~comparison of the residuals of reconstructed positions (i.e., their distances to the real starting positions) is shown in \cref{fig:rasd_map}. The histograms of residuals, as well as residuals of individual coordinates, are shown in \cref{fig:9010_map_res,fig:7030_map_res}. The bin width for the histograms was chosen using the Scott's rule\footnote{The optimal bin width for normally distributed data of size~$N$ with sample standard deviation $\hat{\sigma}$ is $\hat{\sigma}\sqrt[3]{\frac{24\sqrt{\pi}}{N}}$~\cite{scott}.}. As expected, the mean values of the coordinate residuals is zero. The only exception is the $z$\nobreakdash-coordinate in the 70:30 Ar:CO$_2$ track, where the mean is \qty{-0.064}{\cm} (further than one standard deviation \qty{0.051}{\cm} from zero). This is caused by the non-zero initial energy of the ionization electrons (neglected in the other track) which is unaccounted for in the map simulation. Since the primary electron bends towards negative~$z$, the ionization electrons are more likely to be released in this direction (and also bend in the magnetic field towards this direction).
			
			\begin{figure}
				\centering
				\includegraphics[width=0.48\textwidth]{9010_reco_map.png}
				\hfill
				\includegraphics[width=0.48\textwidth]{7030_reco_map.png}
				\caption{Reconstruction (black) of the starting position of ionization electrons (red) using the inversion of the ionization electron map. Two gas compositions 90:10 (left) and 70:30 Ar:CO$_2$ (right) are compared.}
				\label{fig:reco_map}
			\end{figure}
			
			\begin{figure}
				\centering
				\includegraphics[width=0.48\textwidth]{9010_rasd_map.png}
				\hfill
				\includegraphics[width=0.48\textwidth]{7030_rasd_map.png}
				\caption{Comparison of residuals from the \ac{RASD} (black) and map inversion (red) reconstruction methods. Tracks in two different gas compositions 90:10 (left) and 70:30 Ar:CO$_2$ (right) are compared (in the former gas mixture, the drift velocity is higher, therefore the distortion effects are more significant).}
				\label{fig:rasd_map}
			\end{figure}
			
			\begin{figure}
				\centering
				\includegraphics[width=\textwidth]{9010_map_res.png}
				\caption{Map inversion reconstruction residuals (for the individual coordinates and total) of the testing track in 90:10 Ar:CO$_2$ gas mixture.}
				\label{fig:9010_map_res}
			\end{figure}
			
			\begin{figure}
				\centering
				\includegraphics[width=\textwidth]{7030_map_res.png}
				\caption{Map inversion reconstruction residuals (for the individual coordinates and total) of the testing track in 70:30 Ar:CO$_2$ gas mixture.}
				\label{fig:7030_map_res}
			\end{figure}
			
			
		
	\section{Discrete Reconstruction}		
		In order to get a~more realistic representation of a~track measured in the \ac{OFTPC}, we need to take the discretization of the position and time data into account. The readout of the \ac{OFTPC} will consist of 128~pads, their layout is shown in \cref{fig:padlayout}. Time is read out in discrete bins of size $t_\text{bin} = \qty{100}{\ns}$.
		
		As the first approximation, we can neglect the multiplication in the triple\nobreakdash-\ac{GEM} and assume an ideal charge readout. The time measurement starts at the beginning of the electron/positron simulation. The readout coordinates ${(x',y',t)\in\mathcal{R}}$ of each ionization electron can be mapped to the pad coordinates~$(n_\text{pad},n_t) \in \mathcal{P}$:
			\begin{gather}
				n_\text{pad} = n\colon (x',y') \in \left[x_{1,n}-\frac{g}{2},x_{2,n}+\frac{g}{2}\right)\times\left[y_{1,n}-\frac{g}{2},y_{2,n}+\frac{g}{2}\right),\\
				n_t = \left\lceil \frac{t}{t_\text{bin}}\right\rceil,
			\end{gather}
		where $x,y_{1,n}$ and $x,y_{2,n}$ are the opposing pad corner coordinates, and $g$ is the gap between the pads (described in detail in \cref{sec:coor}). This way, the closest pad is assigned to each readout position within the \ac{OFTPC} volume\footnote{Some positions near the wall are not handled and some pads extend beyond the \ac{OFTPC} volume.}. The number of electrons collected by each pad (i.e., collected charge) in each time bin is then counted and serves as a~weight for the energy reconstruction. The reconstructed track consists of points for each $(n,n_t)\in\mathcal{P}$, we get these by reconstructing the position of a~hypothetical electron with the readout coordinates of the pad/time bin center:\footnote{Mapping the center of the pad (along with the midpoint of the time bin) is not necessarily the best approach since it might not correspond to the average parameters of an electron with these readout parameters.}
			\begin{equation}
				\mathcal{D} \ni (x,y,z) = \overline{\mathcal{M}}\left(x_{c,n},y_{c,n},\left(n_t-\frac{1}{2}\right)t_\text{bin}\right).
			\end{equation}
		Examples of positions reconstructed for a~single pad are shown in \cref{fig:pad_reco}. For pads near the magnet poles, the interpolation becomes problematic, due to large distortion. We can use the slower, unoptimized gradient descent search instead, the results are shown in \cref{fig:pad_reco_old}. An example of a~track from the grid-like testing sample described in \cref{sec:microgrid} is shown in \cref{fig:pad_track}
		
		\begin{figure}
			\centering
			\begin{subfigure}[t]{0.48\textwidth}
				\centering
				\includegraphics[width=\textwidth]{pad_66_9010.png}
				\caption{Pad 66 in the 90:10 Ar:CO$_2$ map.}
			\end{subfigure}
			\hfill
			\begin{subfigure}[t]{0.48\textwidth}
				\centering
				\includegraphics[width=\textwidth]{pad_66_7030.png}
				\caption{Pad 66 in the 70:30 Ar:CO$_2$ map.}
			\end{subfigure}
			\begin{subfigure}[t]{0.48\textwidth}
				\centering
				\includegraphics[width=\textwidth]{pad_12_9010.png}
				\caption{Pad 12 in the 90:10 Ar:CO$_2$ map.}
			\end{subfigure}
			\hfill
			\begin{subfigure}[t]{0.48\textwidth}
				\centering
				\includegraphics[width=\textwidth]{pad_12_7030.png}
				\caption{Pad 12 in the 70:30 Ar:CO$_2$ map.}
			\end{subfigure}
			\caption{Reconstruction of detector coordinates for ionization electrons that are read out in a~certain pad (number 66 near the center of the \ac{OFTPC}, and number 12 near the magnet pole that suffers from large distortion effects, causing issues with interpolation on the inverse grid) and time bin. Boundaries of regions mapped to these discrete coordinates are shown in black, reconstructed center of the pad and time bin is shown in red, and map points used for the center reconstruction are shown in blue. Readout is placed at $z = \qty{-8}{\cm}$.}
			\label{fig:pad_reco}	
		\end{figure}
		
		\begin{figure}
			\centering
			\includegraphics[width=0.48\textwidth]{pad_12_9010_old.png}
			\hfill
			\includegraphics[width=0.48\textwidth]{pad_12_7030_old.png}
			\caption{Reconstruction of detector coordinates corresponding to pad 12 (near the magnet pole) for different time bins. Unlike \cref{fig:pad_reco}, we used the gradient descent search, which handles the large distortion better. In its current unoptimized state, these figures took 280 times longer to produce, compared to the interpolation on the inverse grid.}
			\label{fig:pad_reco_old}
		\end{figure}
		
		\begin{figure}
			\centering
			\includegraphics[width=\textwidth]{pads_track.png}
			\caption{Discrete reconstruction of an~\qty{8}{\MeV} track from the grid-like testing sample (described in \cref{sec:microgrid}) with minimal values of $\theta$ and $\phi$ initial direction angles. The simulated ionization vertices are shown in red, the reconstructed track is shown with the colored boxes. The number of electrons in each pad/time bin is denoted with the size of the boxes and their color. The apparent discontinuity is caused by the charge spreading between two pad "columns".}
			\label{fig:pad_track}
		\end{figure}