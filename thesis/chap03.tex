\chapter{Track Reconstruction}
\label{sec:track}
	As the~first step of the~reconstruction algorithm, we reconstruct the~track of a~primary particle -- either an~electron or a~positron. Then, using this information, we determine the~energy of the~particle (Section~\ref{sec:energy}).
	
	The~\textbf{Reconstruction Assuming Steady Drift} uses the~standard \ac{TPC} approach. With parallel fields, the~drift inside a~uniform electric field remains undistorted (as shown in Equation~\ref{eq:drift}). Therefore, we only need to reconstruct the~$z$\nobreakdash-coordinate from the~drift time using the~known drift velocity. We also assume that the~readout coordinates ($x'$,~$y'$,~$t$) are known exactly, neglecting the~pads and time binning.
	
	Reconstruction using an~\textbf{Ionization Electron Map} (from now on referred to as \emph{the~map}) uses a~simulation of the~drift of secondary (ionization) electrons within the~detector volume. This simulation can then be used to interpolate the~initial position of the~secondary electrons. In the~first iteration of this method the~readout is assumed to be continuous.
	
	We present two algorithms using the~map for reconstruction. The~first one uses a~gradient descent algorithm along with trilinear interpolation (see Section~\ref{sec:trilin}) of the~map. The~second method uses interpolation on the~irregular inverse grid with a~polynomial.
	
	The~\textbf{Discrete Reconstruction} uses the~map; instead of reconstructing the~exact position of each electron, we reconstruct the~center of each hit pad together with the~time corresponding to the~midpoint of the~time bin. The~electron count in each \ac{TPC}~bin (consisting of the~pad and the~time bin) serves as an~idealized collected charge, which is then used as a~weight in the~energy reconstruction fit.
	
	\section{Reconstruction Assuming Steady Drift}
	\label{sec:trackfirst}
		As the~first step, we tried to reconstruct a~simulated electron track with a~special set of initial parameters, described in detail in Section~\ref{sec:microfirst}. The~starting point is given by the~origin of our coordinate system and its initial direction is given by the~positive $x$\nobreakdash-axis. This means the~magnetic field of our detector is perpendicular to the~momentum of the~particle at all times, and we can reduce the~problem to two-dimensional space.
		
		For the reconstruction, we decided to use the~common method used in a~standard \ac{TPC} (\red{similar to?}). This will allow us to explore the~significance of the~atypical behavior in our \ac{OFTPC}. Additionally, we assume the~readout is continuous to further simplify the~problem. In this approximation, we reconstruct the~initial position of each ionization electron.
		
		The~reconstruction is then defined by the~following relations between the~coordinates of the~detector space and the~readout space (see Section~\ref{sec:coor}): (\red{some figure, maybe already relating to some track so that it's not too dull})
			\begin{eqnarray}
				x = x',\\
				y = y',\\
				z = v_d t,
			\end{eqnarray}
		where $v_d$ is the~drift velocity of electrons in the~given gas mixture. At a~phenomenological level, this velocity can be considered as a~function of the~electric field~$\bm{E}$ and the~magnetic field~$\bm{B}$ as shown in Equation~\ref{eq:drift}. \red{The~\garfieldpp toolkit uses this fact to accelerate their drift simulation with non-microscopic approaches (could mention in the~simulation chapter).} Since we assume a~uniform electric field in the~detector and in this~approximation we want to neglect the~effect of our unusual magnetic field, we consider the~drift velocity constant. \red{\textbf{Rewrite this while making figures:} We then approximate this velocity by fitting the~dependence $z(t)$ taken from the~simulated ionization electrons. This is in one of the~provisional figures. Also, this description is not completely accurate; in reality, we fit t1:8-y0 with a1*x+a0 and then invert this and use 8-y0 = b1*t1+b0 (old coordinates); b1=1/a1 functions as the~drift velocity. Maybe also define this 8-z variable as an~alternative to z in Section~\ref{sec:coor} and then use it when correcting this.}
		
		\red{Later, in a~commit after this, I plotted some residues (provisional figure), which could be useful, but for some reason they are residuals from a~spline fit of the~track?! Probably redo this without the~spline fit; just explore the~difference in individual points.}
		
		\begin{figure}[H]
			\centering
			\includegraphics[width=0.5\textwidth]{9010_zt.png}
			\caption{Dependence of the~drift time on the~$z$~coordinate in 90~\%~argon and 10~\%~CO$_2$ atmosphere, fitted with a~linear function. The~fitted function gives us the~average drift velocity in the~gas and can be used for rough reconstruction in our \ac{TPC}. \red{Swap for better image with axis labels, etc. Maybe write the~fitted equation.}}
			\label{fig:9010zt}
		\end{figure}
		
		\begin{figure}[H]
			\centering
			\includegraphics[width=0.5\textwidth]{9010_xz.png}
			\caption{The first attempt of a~track reconstruction using only the~drift velocity. This approach works well in a~standard \ac{TPC} (\red{ideally cite some source}). 90~\%~argon and 10~\%~CO$_2$ atmosphere. \red{Swap for better image, correct coordinates.}}
			\label{fig:9010xz}
		\end{figure}
		
		\begin{figure}[H]
			\centering
			\includegraphics[width=0.5\textwidth]{9010_res.png}
			\caption{First attempt at a~track reconstruction using only the~drift velocity, residues. \red{Swap for better image, correct coordinates. What's causing the~shift? Explain details.}}
			\label{fig:9010res}
		\end{figure}
	
	\section{Ionization Electron Map}
	\label{sec:map}
		Inside an~\ac{OFTPC} (\orange{$\exists$ more than one, also considering it a~general concept rather than the specific OFTPC used at this experiment}), the~drift of the~secondary (ionization) electrons is significantly affected by its magnetic field (\red{pictures of the~distortion later, the~effect is bigger for the 90/10 composition.}). We need to take this into account for accurate reconstruction (\red{should be easy to run the~reconstruction without the map and show how much it improves the results}). In the~first approximation, we assume a~continuous readout (i.e., we neglect the~anode segmentation into pads). We can then reconstruct the~original position of each ionization electron using its readout coordinates. For this purpose, we use the~ionization electron map.
		
		The~ionization electron map represents a~mapping from the~detector space to the~readout space (see Section~\ref{sec:coor}). It tells us what readout coordinates $(x',y',t)$ we can expect on average for an~ionization electron created at the~detector coordinates $(x,y,z)$. More precisely, it is a~mapping to the~distributions on the~readout space; we can simplify this as only the means $\overbar{\mathcal{M}}$ (\red{inconsistent notation in the text, write the bar everywhere or nowhere}) and the covariance matrices $\mathcal{M}_\text{cov}$, assuming Gaussian distribution (\red{test this -- some chisq or other statistical test}).
			\begin{equation}
				\overbar{\mathcal{M}}: \mathcal{D} \longrightarrow \mathcal{R},\; (x,y,z) \longmapsto (\bar{x}',\bar{y}',\bar{t}).
			\end{equation}
		To get an~approximation of this mapping, we simulate the~drift of ionization electrons generated on a~regular grid inside the~volume of our \ac{OFTPC}\footnote{The~detector walls are not considered and we simulate the~drift even outside of the~\ac{OFTPC} which allows us to interpolate even close to the~walls}. In order to get accurate results, we use the~microscopic simulation of these electrons described in Section~\ref{sec:microsim} (\red{Monte Carlo from \textit{AvalancheMC} was also considered but it doesn't (didn't?) include magnetic field, we can probably improve this anyway using the fast track simulation with map proposed in the future section}). It is also useful to simulate multiple (100 in our case, \red{this should really only be in a~table since there are two map simulations}) electrons originating from the~same position so that we can account for the~random fluctuations due to collisions.
		
		When evaluating the~map inside the~grid, we use trilinear interpolation (see Section~\ref{sec:trilin}). From now on, we will use the~same symbol $\mathcal{M}$ for this interpolated simulation.
		
		Finally, we need to invert the~map to get the~original detector coordinates $(x,y,z)$ from the~given readout coordinates $(x',y',t)$. In our case, it is reasonable to assume that the mapping $\overbar{\mathcal{M}}$ (\red{of means (notation inconsistency), we lose the information about the~distribution (a~wild idea how to recover this is in the Future section but it will only make sense if the~GEM is already accounted for and is very preliminary as there are many factors to consider)}) is one-to-one (as seen in the simulations). We implemented two methods for this purpose: the gradient descent search (Section~\ref{sec:grad}) and interpolation on the~inverse grid (Section~\ref{sec:interpol}).
		
		The~simulation (\orange{?}) of the~map is a~computationally heavy task. For this reason, we use the~MetaCentrum grid~\cite{metacentrum} to parallelize needed calculations. At first, this was done by evenly distributing the~simulated electrons across the~individual jobs in a~simulation with only one electron per vertex in the~regular grid with a~spacing of one centimeter. Later, a~more efficient approach was implemented, accounting for the~varying lengths of the~drift of individual electrons. If we index the~electrons in the~order of increasing coordinates $y,x,z$ (\red{picture will make things clearer}), we can express the~number~$n_l$ of full XY~layers (i.e., electrons with the~same $z$~coordinate) of electrons with index less than or equal to $i$
			\begin{equation}
				n_l(i) = \left\lfloor\frac{i}{n_{xy}}\right\rfloor,
			\end{equation}
		where $n_{xy}$ is the~number of electrons in each XY~layer calculated simply by counting the~electrons that satisfy boundary conditions for $x$~and~$y$. \red{These conditions should be mentioned above; sector condition + maximal $x$ value.} The~number of electrons remaining in the~top layer is then
			\begin{equation}
				n_r(i) = i\!\!\!\!\mod n_{xy}.
			\end{equation}
		Finally, we can calculate the~sum of the~drift gaps of electrons up to index~$i$
			\begin{equation}
				d_\text{sum} = (z_\text{max}-z_\text{min})n_{xy}n_l-\frac{n_l(n_l-1)}{2}n_{xy}l+n_r(z_\text{max}-z_\text{min}-n_l l).
			\end{equation}
		We then use a~binary search algorithm to find the~maximum index $i$ such that the value of this sum is less than the~fraction $\frac{\text{job id}}{\text{max job id}}$ of the~total sum. This way we obtain the~minimal and the~maximal index of electrons simulated in the~given job.
		\red{The~spacing $l$ should be probably defined above + picture of the~simulating grid (1 layer). zmin zmax also}
		
		\red{After the~simulation of the~map, we calculate the~mean readout coordinates assuming Gaussian distribution (i.e., we use averages). We also calculate standard deviations in a~later commit, should be upgraded to the~covariance matrix. We never actually plotted the~distributions we get when simulating the~same electron multiple times, so we do not know if our assumptions are accurate (could also run some statistical test to see how well the~Gaussian distribution fits).}
		
		\red{The~obtained map is then stored in a~custom class template \textit{Field}, could expand on that. Maybe earlier, since the~same template is used for the~magnetic field.}
		
		\red{Could insert a~table here describing all 4 simulations of the~map (gas composition, spacing, etc.). Simulation inside of one sector (at first double angle). Extra space on the~sensor. Edge cases not taken into account (TPC wall). Using qsub (not sure if important). Add plots of distortion of the~coordinates. Could also do these~plots in a~different way (e.g., drawing all the~endpoints of each ionization electron or some error ellipse plot).}\\
		
		\noindent\red{Images to add (comparison of both simulations):}
		\begin{itemize}[topsep=4pt,itemsep=2pt]
			\item \red{Already have a simple 2D map visualization from the RD51 presentation, can use it or make something better}
			\item \red{3D visualization of the map, simulation example}
			\item \red{$z$ vs. $t$ plot}
			\item \red{XY plane distortion for different $z$ values; with arrows and error bars, for all $z$-layers with different colors}
			\item \red{XZ plane ($y = 0$) distortion in $x$ (maybe not necessary?)}
			\item \red{XT plot ($y = 0$) showing (small) distortion in drift times}\\
		\end{itemize}
		
		\noindent\red{More images:}
		\begin{itemize}[topsep=4pt,itemsep=2pt]
			\item \red{Residuals of the~continuous readout reconstruction.}
		\end{itemize}		
		
		\begin{figure}[H]
			\centering
			\includegraphics[width=0.5\textwidth]{map_9010_gen.png}
			\caption{Example of map generation. \red{Swap for better image, correct coordinates.}}
			\label{fig:map9010gen}
		\end{figure}
		
		\begin{figure}[H]
			\centering
			\includegraphics[width=0.5\textwidth]{9010_reco.png}
			\caption{Example reconstruction with the~map. \red{Swap for better image, correct coordinates.}}
			\label{fig:9010reco}
		\end{figure}
		
		\subsection{Gradient Descent Algorithm}
		\label{sec:grad}			
			The~first implemented method of reconstruction uses a~gradient descent algorithm to calculate an~inversion of the~map $\overbar{\mathcal{M}}$ in a~given point. Gradient descent is an~iterative minimization algorithm for multivariate functions. Let $R\in\mathcal{R}$ be a~point in the~readout space; we want to find a~point $D = (x,y,z) \in\mathcal{D}$ in the~detector space such that 
				\begin{equation}
					\overbar{\mathcal{M}}(D) = R = (x'_R,y'_R,t_R).
				\end{equation}
			We define a~function~$f_R$ in the~readout space as a~distance in this space:
				\begin{equation}
					f_R(x',y',t) = \sqrt{(x'-x'_R)^2+(y'-y'_R)^2+v_d^2(t-t_R)^2},
				\end{equation}
			where $v_d$ is an~approximation of the~drift velocity in the~\ac{TPC}, obtained from the~reconstruction in Section~\ref{sec:trackfirst} (\red{there will be an image with the~linear fit here}). We make an~initial guess (\red{actually in the~original code we just take $z=0$}):
				\begin{equation}
					D_0 = (x'_R,y'_R,v_dt).
				\end{equation}
			Assuming we have the $n$-th estimate $D_n$, we calculate the~$i$-th component of the~gradient of $f_R\circ\overbar{\mathcal{M}}$ numerically using central differences: (\orange{signs look correct})
				\begin{equation}
					\left[\nabla(f_R\circ\overbar{\mathcal{M}})\right]^i(D_n) \approx \frac{f_R(\overbar{\mathcal{M}}(D_n+s\cdot e^i))-f_R(\overbar{\mathcal{M}}(D_n-s\cdot e^i))}{2s},
				\end{equation}
			where $e^i\in\mathcal{D}$ is the~$i$-th coordinate vector and $s$ is the~step size. The~step size should be sufficiently small; initially, we set it as a~fraction $s = \frac{l}{10}$ of the~map's grid spacing $l$. During the~minimization, we check that $f_R(\overbar{\mathcal{M}}(D_n))<10s$ at all times (\orange{$s$ can (?) change -- check}). \red{When using trilinear interpolation, it would be more efficient to calculate the~gradient explicitly ($\pm$ same result). This could be implemented inside the~\textit{Field} template class.} The~next iteration can be calculated as follows:
				\begin{equation}
					D_{n+1} = D_n - \gamma \nabla(f_R\circ\overbar{\mathcal{M}})(D_n),
				\end{equation}
			where $\gamma\in\mathbb{R}^+$ is the~damping coefficient. It should be set to a~small enough value to ensure convergence, but large enough for sufficient converging speed. The~minimization stops either when the error $f_R(\overbar{\mathcal{M}}(D_n))$ drops below a~specified value or when the~number of iterations exceeds a~certain limit (in this case, a~message is printed into the~console).
			The parameters of this method can be further optimized (e.g., a~better choice of $\gamma$, \red{gradient computation}); instead, we later decided to use the~interpolation on the~inverse grid described in the~next section.
			
			\red{Measure reconstruction duration and compare it with the~inverse grid interpolation? Also compare the~result? Typical evolution of $D_n$ during search. Not sure if this has to be cited.}
		
		\subsection{Interpolation on the~Inverse Grid}
		\label{sec:interpol}			
			\red{Interpolation should be generally faster than the~gradient descent since we don't need to iterate. We also don't need to optimize it to improve performance, if it's too slow we can even calculate the~coefficients for the~entire map before reconstruction (again, do some profiling).}
			
			The~best current algorithm uses the~interpolation on the~inverse grid. Rather than inverting the~trilinearly interpolated map using a~numerical minimization method as in the~previous section, we take advantage of the~fact that the~map $\overbar{\mathcal{M}}$ is one-to-one (\orange{isomorphism is supposed to preserve structure, not sure how to interpret that here}, \red{not the best description, we already (kind of) assume it is a~bijection by saying we will invert it}). Since we have simulated values of this map on a~regular grid in the~detector space $\mathcal{D}$, we also know the~inverse map $\overbar{\mathcal{M}}^{-1}$ on the~irregular inverse grid in the~readout space $\mathcal{R}$. To get an~approximation of the~inverse map in the~entire readout space, we can use interpolation (\orange{general concept, the specific choice is described below}).
			
			Since the~inverse grid is irregular, trilinear interpolation cannot be applied. Given that the~simulated map is dense enough to provide a~good approximation considering the~size of our pads, we can adopt a~similar approach.\footnote{A more complicated and computationally heavy alternative would be natural neighbor interpolation or Kriging.} As shown in Equation~\ref{eq:trilinpoly} in Section~\ref{sec:trilin}, trilinear interpolation (\orange{shouldn't need an article when talking about a~general concept}) can be expressed as a~polynomial:
				\begin{equation}
					\widehat{f}(x,y,z) = axyz + bxy + cxz + dyz + ex + fy + gz + h,
				\end{equation}
			where $a,b,c,d,e,f,g,h$ are coefficients uniquely determined by the~values of the~function at the~vertices of the~interpolation cell (\orange{can be calculated in the way shown in the mentioned equation, not sure what more to add}). We can generalize this for a~function defined on an~irregular grid. Given the~function values at any eight points, we can write a~system of eight linear equations
				\begin{equation}
					\begin{pmatrix}
						x_1 y_1 z_1 & x_1 y_1 & x_1 z_1 & y_1 z_1 & x_1 & y_1 & z_1 & 1\\
						\vdots & \vdots & \vdots & \vdots & \vdots & \vdots & \vdots & \vdots\\
						x_8 y_8 z_8 & x_8 y_8 & x_8 z_8 & y_8 z_8 & x_8 & y_8 & z_8 & 1
					\end{pmatrix}
					\begin{pmatrix}
						a\\
						\vdots\\
						h
					\end{pmatrix}
					=
					\begin{pmatrix}
						f(x_1,y_1,z_1)\\
						\vdots\\
						f(x_8,y_8,z_8)
					\end{pmatrix},
				\end{equation}
			which has a~unique solution for the~coefficients for most values of $(x_n, y_n, z_n)$ and $f(x_n,y_n,z_n)$, where $n\in\{1,\ldots,8\}$.
			
			This approach introduces a~small complication: finding the~correct pseudocell (i.e., the image of eight vertices forming a~cubic cell in the~regular grid) in the~inverse grid. The~eight irregularly spaced vertices of this pseudocell do not define a~unique volume, so there are multiple possible ways to partition $\mathcal{R}$ into pseudocells, with no obvious choice among them.
			
			\red{We are currently ignoring this problem and performing binary search along $x$, $y$, $z$ (in this order). It shouldn't matter too much because the~70/30 map doesn't cause such a big distortion and was even accidentally extrapolated for all $z$ different from the central plane.}
			
			\begin{figure}[H]
				\centering
				\includegraphics[width=0.8\textwidth]{interpol.png}
				\caption{Selection of the~points for interpolation. \red{Create better images; use the~explanation interpolation vs. extrapolation strange property. Solution~2 probably does not make much sense.}}
				\label{fig:interpol}
			\end{figure}
		
	\section{Discrete Reconstruction}
		\red{Reconstruction with pads and time bins. Maybe testing different pads.}
		
		\red{It is also possible to make this a~subsection of the~map, making the previous subsections parts of a~new subsection 'Map Inversion'.}
		
		In order to get a~more realistic representation of a~track measured in the~\ac{OFTPC}, we need to take the~discretization of the~position and time data into account. The~readout of the~\ac{OFTPC} will consist of 128~pads, their layout is shown in Figure~\ref{fig:padlayout}. Time is read out in discrete bins of size $t_\text{bin} = 100$~ns.
		
		As the~first approximation, we can neglect the~multiplication in the~triple\nobreakdash-\ac{GEM} and assume an~ideal charge readout. The~time measurement starts at the~beginning of the~electron/positron simulation (\orange{depending on the specific simulation it can correspond to the~production in the target or when entering the OFTPC, here the specific time doesn't matter too much since the primary particle travels basically at light speed (30~ps/cm) which is circa immediate given the time binning}). \red{Randomize this time a bit and see what it does to the~reconstruction.} The~readout coordinates ${(x',y',t)\in\mathcal{R}}$ of each ionization electron can be mapped to the pad coordinates~$(n_\text{pad},n_t) \in \mathcal{P}$:
			\begin{gather}
				n_\text{pad} = n\colon (x',y') \in \left[x_{1,n}-\frac{g}{2},x_{2,n}+\frac{g}{2}\right)\times\left[y_{1,n}-\frac{g}{2},y_{2,n}+\frac{g}{2}\right),\\
				n_t = \left\lceil \frac{t}{t_\text{bin}}\right\rceil,
			\end{gather}
		where $x,y_{1,n}$ and $x,y_{2,n}$ are the~opposing pad corner coordinates, and $g$ is the gap between the~pads (described in detail in Section~\ref{sec:coor}). This way, the closest pad is assigned to each readout position within the~\ac{OFTPC} volume\footnote{Some positions near the wall are not handled and some pads extend beyond the~\ac{OFTPC} volume. \red{This is where an~electric field simulation would come in handy.}}. \red{Makes sense since the~pads attract the electrons, the inhomogeneity of electric field is neglected.} The number of electrons collected by each pad (i.e., collected charge) in each time bin is then counted and serves as a~weight for the~energy reconstruction. The~reconstructed track consists of points for each $(n,n_t)\in\mathcal{P}$, we get these by reconstructing the~position of a~hypothetical electron with the~readout coordinates of the~pad/time bin center:\footnote{Mapping the~center of the pad (along with the~midpoint of the~time bin) isn't necessarily the~best approach since it might not correspond to the~average parameters of an~electron with these readout parameters.}
			\begin{equation}
				\mathcal{D} \ni (x,y,z) = \overbar{\mathcal{M}}\left(x_{c,n},y_{c,n},\left(n_t-\frac{1}{2}\right)t_\text{bin}\right).
			\end{equation}