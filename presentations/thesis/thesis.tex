\documentclass{beamer}
\usetheme{Madrid}
\setbeamertemplate{bibliography item}{\insertbiblabel}

\usepackage[main=english,czech]{babel}
%\usepackage[T1]{fontenc}
\usepackage[utf8]{inputenc}
\usepackage{url}
\usepackage{caption}
\usepackage{graphicx}
\usepackage{xcolor}

\captionsetup[figure]{font=footnotesize, justification=justified, format=hang}

\usepackage[warnundef]{jabbrv}
\usepackage[natbib,style=iso-numeric,sorting=none,backend=biber]{biblatex}
\addbibresource{bibliography.bib}

%%% Detailed settings of bibliography
\ifx\citet\undefined\else

% Maximum number of authors of a single work. If exceeded, "et al." is used.
\ExecuteBibliographyOptions{maxnames=2}
% The same setting specific to citations using \citet{...}
\ExecuteBibliographyOptions{maxcitenames=2}
% The same settings specific to the list of literature
%\ExecuteBibliographyOptions{maxbibnames=2}

% Shortening first names of authors: "E. A. Poe" instead of "Edgar Allan Poe"
%\ExecuteBibliographyOptions{giveninits}
% The same without dots ("EA Poe")
%\ExecuteBibliographyOptions{terseinits}

% If your bibliography entries are hard to break into lines, try this mode:
%\ExecuteBibliographyOptions{block=ragged}

% Possibly reverse the names of the authors with the non-ISO styles:
%\DeclareNameAlias{default}{family-given}

% Use caps-and-small-caps for family names in ISO 690 style.
\let\familynameformat=\textsc

% We want to separate multiple authors in citations by commas
% (while we use semicolons in the bibliography as per the ISO standard)
\DeclareDelimFormat[textcite]{multinamedelim}{\addcomma\space}
\DeclareDelimFormat[textcite]{finalnamedelim}{\space and~}

\fi

%%% Journal Abbreviations (jabbrv package) -- extra custom definitions (following ISO 4)
\DefineJournalPartialAbbreviation{Instrument}{Instrum}
\DefineJournalAbbreviation{Polonica}{Pol}
\DefineJournalPartialAbbreviation{Spectrometer}{Spectrom}

\AtBeginSection[]
{
	\begin{frame}<beamer>[noframenumbering]
		\frametitle{Outline}
		\tableofcontents[currentsection]
	\end{frame}
}

\title[OFTPC track simulation \& reconstruction]{Simulation and Reconstruction of Charged Particle Trajectories in a Time Projection Chamber with Orthogonal Fields}
\author[M.~Vavřík]{\foreignlanguage{czech}{Martin Vavřík}\vspace{0.5cm}\\martin.vavrik@utef.cvut.cz\\MFF UK \& IEAP CTU PRAGUE\\}
\logo{\includegraphics[width=0.08\textwidth]{../images/mff_uk.png}}
\date{September 9, 2025}

\begin{document}
	
	\begin{frame}
		\titlepage
		\flushleft\scriptsize{Funding: GAČR GA21-21801S\\}
		\tiny{Computational resources were provided by the e-INFRA CZ project (ID:90254),\\ supported by the Ministry of Education, Youth and Sports of the Czech Republic.}
	\end{frame}
	
	\begin{frame}
		\frametitle{Outline}
		\tableofcontents
	\end{frame}
	
	\section{Motivation}
	\begin{frame}
		\frametitle{Motivation: ATOMKI Measurements}
		\begin{itemize}
			\item Measurement of anomalies in the angular correlation of an~electron-positron pair internally produced in excited $ {}^8\text{Be} $ and $ {}^4\text{He} $\newline
			\begin{columns}
				\column{0.33 \textwidth}
					\centering
					\begin{minipage}[t][4cm]{\textwidth}
						\centering
						\includegraphics[width=\textwidth]{../images/atomki_detector.png}
					\end{minipage}
					\small{ATOMKI spectrometer.~\cite{atomki_det}}\\ \vspace{0.1cm}
					\tiny{Beam pipe (black), MWPC, $\Delta$E~det. (red), $E$~scintillators (yellow), light guides (blue)}
				\column{0.33 \textwidth}
					\centering
					\begin{minipage}[t][4cm]{\textwidth}
						\centering
						\includegraphics[width=\textwidth]{figures/atomki_be.png}
					\end{minipage}
					\footnotesize{$ {}^8\text{Be} $, $e^{+}e^{-}$ pair angular correlation.~\cite{atomki_be}}
				\column{0.28 \textwidth}
					\centering
					\begin{minipage}[t][4cm]{\textwidth}
						\centering
						\includegraphics[width=0.7\textwidth]{../images/atomki_he.png}\newline
					\end{minipage}
					\footnotesize{$ {}^4\text{He} $, $e^{+}e^{-}$ pair angular correlation.~\cite{atomki_he}}
			\end{columns}
		\end{itemize}
	\end{frame}
	
	\begin{frame}
		\frametitle{OFTPC: Detector Configuration}
		\begin{itemize}
			\item Time Projection Chamber with Orthogonal Fields (OFTPC) -- electric and magnetic field perpendicular (gas mixture Ar/CO$_2$~--~70/30)
		\end{itemize}
		\begin{columns}
			\column{0.47 \textwidth}
				\centering
				\begin{minipage}[t][4.05cm]{\textwidth}
					\centering
					\textcolor{white}{a\,\,}\includegraphics[width=0.82\linewidth]{../images/diagram.png}\newline
				\end{minipage}
				Two out of the six OFTPC chambers.~\cite{x17_utef}
			\column{0.47 \textwidth}
				\centering
				\begin{minipage}[t][4.05cm]{\textwidth}
					\centering
					\includegraphics[width=0.71\linewidth]{../images/diagram2.png}\newline
				\end{minipage}
				OFTPC with a triple gas electron multiplier (GEM) readout.~\cite{x17_utef}
		\end{columns}
	\end{frame}
	\begin{frame}
		\frametitle{OFTPC: Pad layout}
		\centering
		\includegraphics[width=\textwidth]{figures/padlayout.png}
	\end{frame}

	\begin{frame}
		\frametitle{OFTPC: Reasons for Orthogonal Fields}
		\begin{itemize}
			\item No solenoid -- permanent magnets used to generate the field
			\begin{itemize}
				\item Parallel fields difficult to create with permanent magnets
			\end{itemize}
			\item Space constraints -- granularity of the TPC readout limited in order to fit one SAMPA/SRS hybrid in each of the six sectors
			\begin{itemize}
				\item Parallel fields would bend particles parallel to readout, requiring much larger number of pads
				\item These trajectories would extend to more than one sector, requiring alternative architecture of the detector
			\end{itemize}
			\item We will show a similar resolution for significantly lower cost
		\end{itemize}
	\end{frame}
	
	\begin{frame}
		\frametitle{OFTPC: Complications}
		Inhomogeneous magnetic field (simulated using Maxwell~\cite{ansys_maxwell})
		\begin{columns}
			\column{0.42\textwidth}
				\centering
				\begin{minipage}[t][4cm]{\textwidth}
					\centering
					\includegraphics[width = \textwidth]{figures/mag_xy.png}
				\end{minipage}
			\column{0.58\textwidth}
				\centering
				\begin{minipage}[t][4cm]{\textwidth}
					\centering
					\includegraphics[width = 0.95 \textwidth]{figures/9010_xz.png}
				\end{minipage}
		\end{columns}
		\begin{itemize}
			\item The field interferes with the direction of the drift of ionization electrons
			\item Curvature of the track is not constant in this field (deviation from a~circle)
		\end{itemize}
	\end{frame}
	
	\section{Track Simulation}
	\begin{frame}
		\frametitle{Track Simulation}
		\begin{itemize}
			\item Garfield++ used for track simulation
			\begin{itemize}
				\item Primary relativistic particle simulated using the HEED program~\cite{HEED}
				\item Secondary ionization electrons simulated using microscopic tracking (uses equations of motion)
				\begin{itemize}\item Relatively slow (typically 5-30 CPU hours per track), very precise, especially for small structures.  \end{itemize}
			\end{itemize}
			\item Five batches of 9702 tracks with different initial parameters simulated on a grid (MetaCentrum~\cite{metacentrum})
			\begin{itemize}
				\item Electrons and positrons
				\item 11 different energies from 3~MeV to 13~MeV (covers range for $ {}^8\text{Be} $)
				\item 21 different angles~$\varphi$ and 21 different angles~$\theta$ (next slide)
			\end{itemize}
		\end{itemize}
	\end{frame}
	
	\begin{frame}
		\frametitle{Track Simulation}
		\centering
		\large{Spherical angles ($\theta$,$\varphi$) with respect to~$z$}\\
		\small{$\theta$ taken from the equatorial plane $xy$}
		\centering
		\textcolor{white}{aaaaaaaa}\includegraphics[height= 0.61 \textheight]{figures/tpc_ggb.png}\newline
		\small{Diagram of the batch simulation parameters:\\ $\theta\in[-17.1^\circ,17.1^\circ]$, $\varphi~\in~[-16.3^\circ,16.3^\circ]$, $ E_\text{kin.} \in [3,13] $~MeV.}
	\end{frame}
	
	\begin{frame}
		\frametitle{Simulated Track Example (microscopic tracking)}
		\begin{itemize}
			\item Electron track with kinetic energy 8~MeV, $\theta = 0^\circ$ and $\varphi = 0^\circ$
			\item Diffusion less than 1.5~mm in both directions
		\end{itemize}
		\vspace{0.5cm}
		\begin{columns}
			\column{0.33\textwidth}
				\centering
				\includegraphics[width = 0.95 \linewidth]{figures/7030_yz.png}
				Diffusion front view
			\column{0.33\textwidth}
				\centering
				\includegraphics[width = 0.95 \linewidth]{figures/7030_xz.png}
				Electron drift
			\column{0.33\textwidth}
				\centering
				\includegraphics[width = 0.95 \linewidth]{figures/7030_xy.png}
				Diffusion top view
		\end{columns}
	\end{frame}

	\begin{frame}
		\frametitle{Ionization Electron Map Simulation}
		\begin{itemize}
			\item We want an~unambiguous map of the drift of ionization electrons for the reconstruction
			\item We can use a~simulation of evenly spaced electrons
			\begin{itemize}
				\item Current spacing 5~mm, 100 electrons simulated in each location with 0.1~eV energy in a~random direction
			\end{itemize}
		\end{itemize}
		\begin{columns}
			\column{0.06\textwidth}
			\column{0.44\textwidth}
				\centering
				\begin{minipage}[t][4.2cm]{\textwidth}
					\centering
					\includegraphics[width = \textwidth]{figures/7030_xz.png}\\
				\end{minipage}
				{Electron drift}
			\column{0.44\textwidth}
				\centering
				\begin{minipage}[t][4.2cm]{\textwidth}
					\centering
					\includegraphics[width=0.7\textwidth]{figures/map_lines.png}\\
				\end{minipage}
				{Partial simulation of the map}
			\column{0.06\textwidth}
		\end{columns}
	\end{frame}

	\begin{frame}
		\frametitle{Ionization Electron Map Simulation}
		\begin{itemize}
			\item As a result we get an approximation of a mapping from initial coordinates of the electrons $(x,y,z)$ to the readout coordinates $(x',y',t)$
			\item By interpolating we can get the inverse map
			\item We can use the inverse map to finally create mapping from our discrete readout values (channel number, time) to voxels of the primary track
		\end{itemize}
		\begin{figure}
			\centering
			\includegraphics[height=0.4\textheight]{figures/map_lines.png}\\
			\small{Partial simulation of the map}
		\end{figure}
	\end{frame}
	\begin{frame}
		\frametitle{Ionization Electron Map Simulation}
		\begin{figure}
			\centering
			\includegraphics[width=\textwidth]{figures/map_3dvis.png}
			\small{3D visualization of the simulated mapping $\overline{\mathcal{M}}$ and the inverse mapping $\overline{\mathcal{M}}^{-1}$.}
		\end{figure}
	\end{frame}
	\begin{frame}
		\frametitle{Ionization Electron Map Simulation}
		\begin{figure}
			\centering
			\includegraphics[height=0.68\textheight]{figures/map_yx_all_7030.png}\\
			\small{$x$ and $y$ coordinate distortion at different $z$ values (denoted by colors).}
		\end{figure}
	\end{frame}

	\section{Track Reconstruction}
	\begin{frame}
		\frametitle{Track Reconstruction}
		\begin{itemize}
			\item At first using only the inverse map (not accounting for readout pads)
			\item Later simple reconstruction with pads and time bins, counting the number of electrons in each bin
		\end{itemize}
		\begin{columns}
			\column{0.42\textwidth}
				\centering
				\begin{minipage}[t][4.5cm]{\textwidth}
					\centering
					\vspace{0.5cm}
					\includegraphics[width=\textwidth]{figures/7030_reco_map.png}\\
				\end{minipage}
				\small{Original and reconstructed interaction points on the simulated track}
			\column{0.58\textwidth}
				\centering
				\begin{minipage}[t][4.5cm]{\textwidth}
					\centering
					\includegraphics[width=\textwidth]{figures/pads_track.png}\\
				\end{minipage}
				\small{Reconstruction with pads}
		\end{columns}
	\end{frame}
	\begin{frame}
		\frametitle{Inverse Mapping of Pads}
		\begin{columns}
			\column{0.06\textwidth}
			\column{0.44\textwidth}
			\centering
			\begin{minipage}[t][4.2cm]{\textwidth}
				\centering
				\includegraphics[width = \textwidth]{figures/pad_12_7030_old.png}\\
				\vspace{-0.5cm}
				{Pad 12 (near the magnet pole)}
			\end{minipage}
			\column{0.44\textwidth}
			\centering
			\begin{minipage}[t][4.2cm]{\textwidth}
				\centering
				\includegraphics[width = \textwidth]{figures/pad_66_7030.png}\\
				\vspace{-0.5cm}
				{Pad 66}
			\end{minipage}
			\column{0.06\textwidth}
		\end{columns}
	\end{frame}

	\section{Energy Reconstruction}
	\begin{frame}
		\frametitle{Energy Reconstruction}
		\begin{itemize}
			\item Prefit with circle with smoothly attached lines
			\item Kinetic energy fit using 4\textsuperscript{th} order Runge-Kutta
			\item Known initial position and direction of the particle assumed
			\item Currently cca 0.3~CPU seconds per track
		\end{itemize}
		
		\centering
		\includegraphics[width=0.9\textwidth]{figures/cfit3d_track.png}\\
	\end{frame}
	\begin{frame}
		\frametitle{Energy Reconstruction Precision}
		\centering
		\begin{columns}
			\column{0.5\textwidth}
			\centering
			\Large \textbf{Electrons}
			\begin{figure}
				\centering
				\includegraphics[width = 0.95 \linewidth]{figures/rk_e_dE.png}
			\end{figure}
			\column{0.5\textwidth}
			\centering
			\Large \textbf{Positrons}
			\begin{figure}
				\centering
				\includegraphics[width = 0.95 \linewidth]{figures/rk_p_dE.png}
			\end{figure}
		\end{columns}
		\vspace{0.5cm}
		\footnotesize{Relative reconstruction deviation of the~kinetic energy of electron and positron tracks (cca 24\,000 of each simulated).}
	\end{frame}
	\begin{frame}
		\frametitle{Energy Reconstruction Precision}
		\centering
		\begin{columns}
			\column{0.5\textwidth}
			\centering
			\Large \textbf{Electrons}
			\begin{figure}
				\centering
				\includegraphics[width = 0.95 \linewidth]{figures/rk_e_dE_E.png}
			\end{figure}
			\column{0.5\textwidth}
			\centering
			\Large \textbf{Positrons}
			\begin{figure}
				\centering
				\includegraphics[width = 0.95 \linewidth]{figures/rk_p_dE_E.png}
			\end{figure}
		\end{columns}
		\vspace{0.5cm}
		\footnotesize{Relative reconstruction deviation of the~kinetic energy of electron and positron tracks (cca 24\,000 of each simulated).}
	\end{frame}
	\begin{frame}
		\frametitle{Energy Reconstruction Precision}
		\centering
		\begin{columns}
			\column{0.5\textwidth}
			\centering
			\Large \textbf{Electrons}
			\begin{figure}
				\centering
				\includegraphics[width = 0.95 \linewidth]{figures/rk_e_dE_ph.png}
			\end{figure}
			\column{0.5\textwidth}
			\centering
			\Large \textbf{Positrons}
			\begin{figure}
				\centering
				\includegraphics[width = 0.95 \linewidth]{figures/rk_p_dE_ph.png}
			\end{figure}
		\end{columns}
		\vspace{0.5cm}
		\footnotesize{Relative reconstruction deviation of the~kinetic energy of electron and positron tracks (cca 24\,000 of each simulated).}
	\end{frame}
	\begin{frame}
		\frametitle{Energy Reconstruction Precision}
		\centering
		\begin{columns}
			\column{0.5\textwidth}
			\centering
			\Large \textbf{Electrons}
			\begin{figure}
				\centering
				\includegraphics[width = 0.95 \linewidth]{figures/rk_e_dE_th.png}
			\end{figure}
			\column{0.5\textwidth}
			\centering
			\Large \textbf{Positrons}
			\begin{figure}
				\centering
				\includegraphics[width = 0.95 \linewidth]{figures/rk_p_dE_th.png}
			\end{figure}
		\end{columns}
		\vspace{0.5cm}
		\footnotesize{Relative reconstruction deviation of the~kinetic energy of electron and positron tracks (cca 24\,000 of each simulated).}
	\end{frame}
	
	
	\section{Summary \& Future}	
	\begin{frame}
		\frametitle{Summary}
		\begin{itemize}
			\item Several batches of tracks have been simulated for testing purposes.
			\begin{itemize}
				\item $\theta\in[-17.1^\circ,17.1^\circ]$, $\varphi~\in~[-16.3^\circ,16.3^\circ]$, $ E_k \in [3,13] $~MeV
			\end{itemize}
			\item The map of secondary electron positions and drift times has been generated.
			\item The map has been tested by the track reconstruction.
			\item The results suggest that:
			\begin{itemize}
				\item Current energy resolution (FWHM) is 1.6~\% for electrons and 2.0~\% for positrons.
				\item OFTPC works well on a simulation level.
				\item Some of the systematic errors can be corrected.
			\end{itemize}
		\end{itemize}
	\end{frame}
	\begin{frame}
		\frametitle{Future Steps}
		\begin{itemize}
			\item Account for parasitic tracks caused by high energy secondary electrons
			\item Account for GEM in the simulation, charge distribution between pads
			\item Account for noise (denoising using a convolutional neural network --- M. Gajdoš~\cite{Gajdoš_2025})
			\item Optimize Runge-Kutta integration fit with likelihood approach (instead of least squares) if needed
			\item Implement a faster simulation method for ionization electrons using the map
			\item Test different geometries
			\item Make a better simulation of magnetic and electric fields, compare with measurements
		\end{itemize}
	\end{frame}
	
	{
		%\usebackgroundtemplate{\includegraphics[width=\paperwidth,height=\paperheight]{../images/DSC_5602.jpg}}%
		\begin{frame}[noframenumbering]{}
			\begin{center}
				\Huge Thank you for your attention.
			\end{center}
		\end{frame}
	}
	
	%\section{References}
	\begin{frame}[allowframebreaks,noframenumbering]
		\frametitle{References}
		\printbibliography
		%\bibliography{references}
		%\bibliographystyle{unsrt}
	\end{frame}
	
	\begin{frame}
		\frametitle{Ionization Electron Map Simulation}
		\centering
		\begin{minipage}[c]{0.9\textwidth}
			\begin{figure}
				\centering
				\includegraphics[height=0.7\textheight]{figures/map_yx_bottom_7030.png}\\
				\small{Worst case $x$ and $y$ coordinate distortion for maximal initial distance from readout. Ellipses  with 95~\% confidence are shown for each starting position.}
			\end{figure}
		\end{minipage}
	\end{frame}
	\begin{frame}
		\frametitle{Ionization Electron Map Simulation}
		\centering
		\begin{minipage}[c]{0.9\textwidth}
			\begin{figure}
				\centering
				\includegraphics[height=0.7\textheight]{figures/map_xyt_7030.png}\\
				\small{Mapping of the layer furthest from the OFTPC readout.}
			\end{figure}
		\end{minipage}
	\end{frame}
	
	\begin{frame}[noframenumbering]
		\frametitle{Reconstruction Testing Summary}
		\begin{enumerate}
			\item Ionization electron map simulation of secondary electrons in the entire volume of the OFTPC
			\item Track(s) simulated using HEED and microscopic tracking
			\item Counting the number of secondaries in each pad and time bin of the readout
			\item Using the map to reconstruct the position of centers of pads for given centers of time bins
			\item Fitting of the reconstructed points with circle and lines fit using least squares weighted by the number of secondaries in each point
			\item Using the magnetic field in the middle of the track to get first energy estimate
			\item Using the 4\textsuperscript{th} order Runge-Kutta fit with least squares to refine the energy estimate
		\end{enumerate}
	\end{frame}
	
	\begin{frame}[noframenumbering]
		\frametitle{Figure 3.3}
		\centering
		\includegraphics[width=0.48\textwidth]{figures/rk_micro.png} \hfill
		\includegraphics[width=0.48\textwidth]{figures/rk_micro_res.png}\\
		A comparison of the HEED track from the microscopic simulation in Sec.~3.1.1 with a~Runge-Kutta track with the same initial parameters and $\tau_\text{step} = 0.1$~ps (reducing the step further doesn't make a~visible difference). In the view of the tracks on the left, the distance of the HEED ionization electrons from the RK4 track is exaggerated $1000\times$. On the right, the dependence of the HEED electrons residuals (i.e., their shortest distance to the RK4 track) on their $z$\protect\nobreakdash-coordinate is shown.
	\end{frame}
	
	
	\begin{frame}[noframenumbering]
		\frametitle{Electron-Positron Internal Pair Creation}
		\centering
		\includegraphics[width=0.75\textwidth]{../images/X17_sim.png}\\
		\begin{minipage}[t][4cm]{0.85\textwidth}
			Histogram of e$^+$e$^-$ pair angle and positron energy produced by standard IPC and by the decay of a hypothetical boson X17 (branching ratio $3\cdot10^{-3}$).
		\end{minipage}
	\end{frame}
	
	\begin{frame}[noframenumbering]
		\frametitle{Electron-Positron Internal Pair Creation}
		\centering
		\includegraphics[width=0.75\textwidth]{../images/X17_sim_rand.png}\\
		\begin{minipage}[t][4cm]{0.85\textwidth}
			The same histogram after accounting for predicted reconstruction errors.
		\end{minipage}
	\end{frame}
	\begin{frame}[noframenumbering]
		\frametitle{Ionization Electron Map Reconstruction Residuals}
		\centering
		\includegraphics[width=0.43\textwidth]{../images/c_fit_res_x}
		\hfill
		\includegraphics[width=0.43\textwidth]{../images/c_fit_res_y}\\
		\includegraphics[width=0.43\textwidth]{../images/c_fit_res_z}
	\end{frame}
	\begin{frame}[noframenumbering]
		\frametitle{Track Simulation -- TPC Window}
		\centering
		\includegraphics[width=0.6\textwidth]{../images/tpc_window_sim.png}
	\end{frame}
	
\end{document}